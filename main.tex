%%%%%%%%%%%%%%%%%%%%%%%%%%%%%%%%%%%%%%%%%%%%%%%%%%%%%%%%%%%

%% document class
\documentclass[11pt,a4paper]{book}
%% packages
\input{settings/packages}

%% page settings
\input{settings/page}
%%%%%%%%%%%%%%%%%%%%%%%%%%%%%%%%%%%%%%%%%%%%%%%%%%%%%%%%%%%
\newcommand{\imginput}[1]{\input{#1}} %this command is to prevent pdf_latex imgs from showing up in structure
\newcommand{\tabinput}[1]{\input{#1}} 
\pdfsuppresswarningpagegroup 1
\begin{document}

\frontmatter
%-------------------------------------------------------------------------------%
%-------------------------------------------------------------------------------%
%-------------------------------------------------------------------------------%
\input{title_page}

\tableofcontents

\chapter{Preface}
	\label{chp:preface}Currently, much of the material being taught today in university and college class are extremely out of touch with what is relevant with today's current research, in physics especially. This has made it difficult for students nearing the end of their studies to jump into todays research projects without getting completely lost. In the physical science field, research projects have become increasingly more complex, technologically and theoretically wise. Long gone are the days of throwing in raw material and hoping for the best, boring titrations, or smoothening metal surfaces by hand, etc etc. This book is aimed at helping young university students jump into laser spectroscopy to aid in acquisition of the necessary theoretical and technical skills. This is just a band-aid solution however, but will provide as a base for my long-term future goal of updating the material being taught in elementary school all the way to the university/college level. 
	
	Another result of todays outdated teaching material is the large gap in experience and knowledge between todays professors and todays students, particularly again in physics. To much pretentious language and jargon. This large gap has made it difficult for professors to effectively communicate complicated concepts and techniques to students as they have long forgotten what it is like to be young and naive. Much of today's papers are also targeted at todays professors completely allienate todays young and inexperienced students. This is not helping at all.
	
	An examples of very complicated AND critical technique for many laser cavity setups is the infamous Pound Drever Hall locking technique. Many physics students(pretty much all but me) get completely lost with this technique because of absent knowledge of material this technique is based off of. Feedback control theory, modulation and heterodyning is absolutely essential for PDH locking, but is not being taught to many students at any point during their studies. It should be at the very least, be given as pre-reading. I plan to remedy this problem by personally over hauling university/college optics class to include locking of a cavity in general. Professors have essentially progressed many fields, not just laser spectroscopy or cold molecules, significantly and thus actually do not know how to effectively help and guide students in learning complex experiments. This book should explain concepts not currently explained to students from their studies so that they may understand this complex field. This is by no means easy and there is a reason why the content taught today is outdated.
	
	Formerly being one of those poor lost students and now one of the leaders of the field, I have gone on to solve the first solution to the cavity problem, create a new cavity aligning technique(used all over the world), and created a systematic method of aligning a laser cavity in the infrared red region resulting in the alignment of the world's first 0.98cm (~1m) mid-infrared cavity. I did all this while still in my early 20s during my undergrade mind you. This book currently serves somewhat as a personal journal of everthing I have learned so as to provide a source at picking at my brain to help communicate everything I needed to learn in order to understand this complex interdisciplinary field. I have decided to make this book free for life on my github(or something else) repository so that anyone can freely view this book. I also will make multiple jupyter notebook page to accompany this book for interactivity with this book as some concepts are just better explained with interactive plots etc etc and images. The jupyter notebook will also serve to help young chemists and physicists learn python as I will comment the crap out of everything.
	
	At some point, I also would like python (more broadly, programming) to be taught teenagers as programming has proven to be a valuable tool in ALL FIELDS. Due to Python's easy to use and learn syntax, vast open source resource library of packages,  generous funding of various groups and societies, it has proven to become a high class, flexible, powerful and universal tool in many fields. Personally, being a chemist, physicist and soon engineer, I will just show be showcasing its uses in these fields, but of course, it has applications in statistic, finance, mathematics, and data science in general etc etc. I am aware of Julia, but I am accustomed to python and am not planning to switch languages again.
	
	I still have much to learn, so this book(formerly a thesis) will continue to grow as I work on my PhD. A lot of this content will also end up in my more professional PhD thesis that I will hand in for my graduate studies. This final thesis will be a professional document(for the snobs). I admit, I have torrented and never financialy contributed anything during my studies, so this will serve as my way of giving back to the academic community. I hope you old farts are ready. I also will not be posting experimental results like you would see in a typical thesis (hence why I call this a book and a separate entity). I will however include experimental data to show the tricks that I use to automate mundane and painfully long data processing. I also will showcase my programs that I created. I am not officially a software engineer yet and have very little knowledge of optimization so be gentle. Programming wise, I am just concerned with them working, being completely automated, and producing reliable results.
	
	Before I start the first chapter, I am going to state various questions to keep in mind when reading so that the most important objectives are clear. Why am I studying chemistry, physics and engineering? Why so many different topics? How did I learn all these things? Why should you read this book? What if you are not interested in lasers, spectroscopy or engineering? The answer to all these questions is because you want to better yourself, you want to love and improve your work, you want to push things as far as {\bfseries YOU} can, or get the job done(at the very least). This book is going to be about laser spectroscopy, particularly employed with NICE-OHMS and Freq combs, but I will break every topic down into their core components. If you are in chemistry, physics, engineering or even math, you {\underline{will}} find many of these core topics and concepts reoccurring in your own field.

\mainmatter
%------------------------------------------------------------------%
%------------------------------------------------------------------%
%------------------------------------------------------------------%
\chapter{Introduction to Laser Spectroscopy}
	\label{chp:Introduction to Laser Spectroscopy}
	Read the preface first please. It is important. I understand it sounds like a journal article or a long rant, but it contains useful knowledge and motivation for this book.
	
	The focus of the very first chapter is to provide a quick and basic run down what to expect from this book. In other words, what I have learned from my studies and how I entwine all these various topics together. There will be no discussion of math or programming in this current chapter. Math will be encountered later on when we dive into the various topics as well as within the Jupyter notebooks. How to completely utilize python will not be covered in this book as the tutorials are complete and simple enough to understand (seriously though, if you can't go through the tutorial you are whiny)\todo{edit later}. There are no shortcut in learning how to program as learning to code takes time, patience and practice. If programming and math are not a major concern in your field, just skip the equations and coding to shift your focus on to the images and plots. As a chemist, I know many of my peers are mathematically challenge and that is perfectly acceptable. For my fellow physicists, stop judging them because all you are spatially handicapped. Grow up engineers.
	
	The focus of my studies so far is centralized about molecular {\bfseries spectroscopy}, linear and nonlinear {\bfseries optics}, electrical and software {\bfseries engineering}. To tie in the connection between spectroscopy and optics, expect to see an enormous amount of {\bfseries quantum mechanics} from both the physics and chemistry interpretations. In addition, expect to see lots of data manipulation with python3.5 (maybe some Igor), instrumental control with Labview, data manipulation and plot interaction an creation within various jupyter notebooks. I will notify which figures contain objects produced within a jupyter notebook. The jupyter notebooks used to create figure images can be located in the images directory within their respective chapters that the images are found in. The Jupyter notebook are heavily commented and contain all the code for making the plots should you want to learn from them or use them yourself. By simulations, I just mean mathematical calculations with adjustable parameters. Just a clarification, simulation are not necessarily just videos of objects moving; this actually took me a while to comprehend.
	
	include a screenshot of jupyter notebook simulation at various states
	\section{Spectroscopy Intro}
		\label{sec:Spectroscopy Intro}
		\begin{figure} [!ht]
			\centering
			\def\svgwidth{\columnwidth}
			\huge
			\resizebox{8cm}{!}{\imginput{images/molecule.pdf_tex}}
			\caption{This is a molecule consisting of two atoms for those who are, for some reason, not acquainted with what molecules. If you do not know what a molecule is, you should not be reading this book.}
			\label{fig:molecule-alpha}
		\end{figure}
		
		{\bfseries Spectroscopy} is the study of matter where techniques are employed to collect their spectral information. Such techniques include the observation of derived ions via the use of a generated and well defined electric field (REMPI) or just interaction of matter with electromagnetic radiation(is mass s). Spectroscopy is important as it has allowed us to verify and interpret many of the different phenomenons predicted by quantum mechanics. Molescules possess unique spectral properties and characteristics which hass enabled us to characterize the structure and properties of molecular species. 
		
		This can be shown through comparing the laser spectrums of
		the isotopes of methane; CH4,CH3D, CH2D2, CHD3, and CD4. Upon examination, some peaks present in all or some of CH3D, CH2D2 and CHD3 spectrums, but will be absent in CH4 and CD4 spectrums. The overall difference between isotopes are "small" relative to comparing different species though, but with high enough resolution, we are able to distinguish the difference isotopes through the spectral properties. The spectral dependencies can be accurately predicted through on the various levels of structure within atoms and molecules which are discussed subsequently.
		
		Remember that molecules consists of elements and it is actually impossible to "photograph" them so spectroscopy is our best method of probing at the atomic levels. The method of interaction discussed in this book typically include emission, absorption and dispersion of electromagnetic radiation with matter and only strongly occur when the frequency of the laser beam is resonant with a transition of the atom or molecule. Transitions cause atoms and molecules to undergo a changes in their initial state[structure] of the wavefunctions[molecule] to another state[structure]. These transitions typically involve energy changes through absorption and/or emission of electromagnetic radiation of specific frequencies. The method for determine which transition are allowed and disallowed will be discussed later on. 
		
		When determining the frequency range of interest, the atom or molecule and the type of transition must be considered so that the appropriate type of laser and detector is used in the experiment. This is important as different types of transitions have different degrees of coupling as well as differences in their nature. For example, electronic transitions involve changes in electronic energy of the electrons with frequency ranging from visible light to ultraviolet. Rovibrational and rotational transitions requires electromagnetic frequencies from the infrared and microwave region respectively. In this book, we will be focusing on rovibrational spectroscopy of molecules.
	
		\begin{figure} [!ht]
			\centering
			\def\svgwidth{\columnwidth}
			\resizebox{15cm}{!}{\imginput{images/abs-rovib-trans.pdf_tex}}
			\caption{Many of the molecules from the  diatomic species above have their vibrational mode  excited into its first vibrational excited state. This is caused by the absorption of the photon forming the laser beams. The effects is the energy from the photons, $qh\nu$, are transfered to the molecules resulting in stronger vibrations of the molecules.}
			\label{fig:abs-rovib-trans}
		\end{figure}	
		
		With molecules, we cannot make the assumption that the nucleus and/or electron within a molecule are non interacting like is done for a large part atomic spectroscopy. In molecules, atom are in such close proximity of each other when bonding such that these interaction must be taken into consideration in order to obtain reliable results. Some of the results of these interactions are shifts in energy level of states due to mixing, various molecular conformations, their associated microenvironments, coupling of modes of something something something (forgot the jargon) addition of rovibrational structure, and (talk about symmetry, man i really gotta brush up on this, super disgraceful). 
		
		
		Heavily edit the paragraph below, got lazy.
		
		I am particularly focused on rovibrational transitions of molecules meaning I will specialize in the use of infrared lasers. Rovibrational spectroscopy strengths lie on its ability to gives structural, isotopic, and conformational information of the molecule but it comes at the cost of having to use infrared. Infrared is particularly difficult to work with as it is invisible to the naked eye thus making it very dangerous to work with. You might blind yourself or others when working with infrared so it is VITAL TO TAKE extreme precaution when aligning with infrared. It is better to be safe then sorry. One particular advantage that I am focused on is its potential for high resolution spectroscopy because of the naturally low energy nature. There is very little development in this field as of this moment meaning there is a lack of good sources of radiation and detection. This is something I wish to change throughout my career. The next target is rotational spectroscopy which should be many times more impossible than rovibrational spectroscopy.
		
		\begin{figure} [!ht]
			\centering
			\def\svgwidth{\columnwidth}
			\Huge
			\resizebox{10cm}{!}{\imginput{images/laser-gaussian-beam.pdf_tex}}
			\caption{Notice how the beam begins to diverge the further it travels away from the laser source. Lasers do not travel in straight lines because of an uncertainty principle.}
			\label{fig:laser-gaussian-beam}
		\end{figure}
		
	\section{Introduction to Lasers and Optics}	
		\label{sec:Introduction to Lasers and Optics}
		So why are lasers so interesting? Why not broadband/divergent sources such as lamps? The  big reason for choosing to work with lasers is because the potentials lasers have for ultrahigh resolution spectroscopy. The techniques involved in creating and utilizing lasers are vastly more costly then using broad band sources such as lamps. Which is better? That is answered entirely on the scope and application of your purposes. The goal for field at which this book is written for, CEAS, is to create highly refined spectrums enough to observe hyperfine structure. 
		
		Hyperfine structure results from interaction of nuclei and electrons within atoms and molecules, but is minuscule in comparison to rovibrational structure. Observation of hyperfine structure is not achievable with broadband sources of radiation because of how wide their linewidths are in comparsion to lasers; more on this later. Lasers have naturally lower linewidths allowing for production of higher resolution spectrums as fine structures may be resolvable. Notice how I say can and maybe. Divergent radiation sources have much larger linewidths making them near impossible to observe hyperfine structure (watch someone prove me wrong). 
		
		\begin{figure} [!ht]
			\centering
			\def\svgwidth{\columnwidth}
			\resizebox{12cm}{!}{\imginput{images/linewidths-gaus-blackbody.pdf_tex}}
			\caption{The red line is a gaussian beam with a frequency centered at about $5.0 \times 10 ^ {14}$ Hz  (600nm) while the blackbody radiation broadened over a large spectrum. The blackbody radiation source can be thought of as tungsten lamp, a broadband source. The frequency of the laser is very well defined in comparions to the black body radiation source. This well defined frequency nature is what enables high resolution spectroscopic details to be extracted from a sample.}
			\label{fig:linewidths-gaus-blackbody}
		\end{figure}
				
		Despite this drawback, divergent radiation sources still have important applications where speed is essential, after all FTIR is still dope. Depending on which is more important, speed or resolution, the appropriate radiation source must be chosen. Lasers nowadays are much more affordable and reliable now a days because of the advances in their technology (need to read up on lasers). Now is a good time to introduce the essential topics of lasers to classroom fields of study, not just in physics and chemistry. It is not necessary to go into full details about their derivation of course as many mathematical tools are essential in the derivation such as vector calculus.
		
		For the topics in optics(thanks Emily), 
		I will mostly cover linear optics in this book and the results an effects of non-linear optics from special crystals. More specifically, I will mostly focus on the effects of using the crystals, as the actual interaction between the crystal and laser beam is quite mathematically challenging (even for most physicists). Nonlinear optics is an important topic however as non-linear crystals have allowed the implementation of many more advance techniques such as modulation, heterodyning and tuning of lasers. You do not need to understand what I just said about non-linear crystals as I am just introducing them. Non-linear effects will largely be overlooked and taken for granted in this book. 
		
		\begin{figure} [!ht]
			\centering
			\def\svgwidth{\columnwidth}
			\resizebox{16cm}{!}{\imginput{images/non-linear-crystals.pdf_tex}}
			\caption{Examples of the modifications that can be performed on an input laser beams. These processes tend to be very inefficient in their conversions as the resulting output laser beams tend to be much lower in intensity than the input laser beams}
			\label{fig:non-linear-crystals}
		\end{figure}
		
	\section{Programming}
		\label{sec:Programming}
		probably\todo{BAH!}\ consist of two parts.
		
		one to talk about initializing instruments and using basic labview techniques to simply control instrument 
		
		another using python3.5 to for data processing
		most likely refer to a jupyter notebook for this
		
		Ignore this section for now, this is just a placeholder for my thoughts. I need to actually experiment more and do some mroe reading on instrumentation and Labview. Computers and associated software have also seen large advances making controlling lasers and other instrumental devices vastly simplified as new devices can simply be controlled by USB where the signal transfer is parallel. It also helps when all the makers of instruments follows the same up to date standards. (I had to use serial communication with instrument older than myself for most of my instruments in my first laser cavity project. It was painfully buggy and made synchronization nearly impossible since with serial, you can only send information one bit at a time. Probably buggy because of the older computers inside the instruments.

	\section{Basic Direct Absorption Spectrosopy Setup}
		\label{sec:Basic Direct Absorption Spectrosopy Setup}
		With all these topics of spectroscopy, optics and engineering in mind, we want to build a simple spectroscopy experiment to collect an absorbance signal from the sample where everything is automated. By automating as much of a experiment as possible, we are reducing human error while freeing up as much time as possible for the experimentalists so that they  can focus on performing more complex tasks that require flexibility and creativity. The setup must be built so that anyone can put in the analyte, turn on your laser and automatically begin data collection followed by data analysis to produce your lovely spectrum(or signal) and extraction vital information.
		
		To illustrate this, direct absorption spectroscopy (useful only in a learning environment) figure will be used as the model in this chapter. To follow \autoref{fig:dir-abs-spec-setup}, start at the laser source and follow the laser and signal all the way to the computer with the spectrum while reading the annotations in order. I will not be going to in dept as the later chapters will bind everything together.	
		
		\begin{figure} [!ht]
			\centering
			\def\svgwidth{\columnwidth}
			\resizebox{16cm}{!}{\imginput{images/dir-abs-spec-setup.pdf_tex}}
			\caption{This image illustrates a basic direct absorption spectroscopy setup with all the essential components.}
			\label{fig:dir-abs-spec-setup}
		\end{figure}	
		
		\noindent
		{\bfseries (A)} The optical system is used to adjust the beams various parameter such as intensity, phase and frequency.\\
		{\bfseries (B)} At least two mirrors must be used to perform fine tuning of the laser in 3 dimensions. This is necessary in order to propagate the beam through the sample and detector crystal. \\
		{\bfseries (C)} As the beam travels through the optical system, there is power loss at various point of the laser beam. Substantial loses occur at the optical system. At\\ {\bfseries ***}, the is beam split so that the a laser intensity can be approximated. The most important location of power loss is at our sample as that is where our signal lives.\\
		{\bfseries(D)} The detector analog signals travels to the USB-DAQ with some modification to the signal if necessary. The USB-Daq then converts the analog signals into digital so that can be transferred to a computer to be stored and processed. \\
		{\bfseries(E)} More than one program can be used in the data collection and processing. For example, Labview can be used to initiate communication with the instruments, such as the laser, and initiate the USB-Daq for data acquisition. The generated data file can be loaded on to IPython console, Jupyter notebook, or Igor for automated data manipulation and plot creation.		
	
	\section{Advance Techniques}
		\label{sec:Advance Techniques}
		The rest of the content will begin to increase in complexity as we focus on more relevant spectroscopy techniques such as frequency modulation spectroscopy, Pound-Drever-Hall locking technique, cavity enhanced absorption spectroscopy, frequency combs, noise immune cavity enhanced optical heterodyne molecular spectroscopy (why Dr. Ye....), and something something.

%-------------------------------------------------------------------------------%
%-------------------------------------------------------------------------------%
%-------------------------------------------------------------------------------%
\chapter{Fundamentals of Spectroscopy}
	\label{chp:Fundamentals of Spectroscopy}
	
	\begin{figure} [!ht]
		\centering
		\Large
		\def\svgwidth{\columnwidth}
		\resizebox{13.5cm}{!}{\imginput{images/particle-wave-duality.pdf_tex}}
		\caption{{\bfseries Particles} behave just like any old soccer ball. They collide and reflect off surfaces \newline
			{\bfseries Waves} can diffract, become attenuated in intensity, and interfere with each. \newline
			{\bfseries Microscopic particles} such as electrons and photons show both characteristics. \newline
			Also, all electrons are blue because Budha made them that way.}
		\label{fig:particle-wave-duality}
	\end{figure}
	
	Spectroscopy is the field where atoms and molecules are experimentally probed to obtain spectral information for characterization. There are actually a number of creative methods and techniques currently employed to obtain spectroscopic detail of the target species such as Resonance-Enhanced MultiPhoton Ionization, but this book will focus on the class of Laser Absorption Spectroscopy (\autoref{sec:Laser Absorption Spectroscopy}) techniques for observations \textbf{possible} transitions of quantum systems. With frequency ranges in the infrared to microwave, rovibrational and rotational transition of targeted molecules or radicals are observed. Rovibrational spectrums collected from these target species will provide details such as the bonding order between the elements, conformational structure of the molecules and isotopic information of the elements. 
	
	Before jumping into infrared spectroscopy, it is necessary to discuss a bit of quantum mechanics and how molecules interact with electromagnetic radiation \autoref{fig:photon-electron-transition} in order to lay out the foundation for understanding spectroscopy. The proofs and derivation will largely be simplified since this is not a quantum mechanics textbook. The classical electromagnetic radiation will also now be treated as the quantum mechanical photon in this chapter out of convenience. 
	
	\begin{figure} [!ht]
		\centering
		\Large
		\def\svgwidth{\columnwidth}
		\resizebox{12cm}{!}{\imginput{images/methyl-radical.pdf_tex}}
		\caption{{Methyl Radical (CH3) radical. It is essentially methane (CH4), a tetrahedral molecule, but is missing a hydrogen bond. It is radical since it has a lone electron. It is also neutral in charge. I probably need to learn how to use molecular structure program. Or just get someone else to remake this and create all future molecules for me.}}
		\label{fig:methyl-radical}
	\end{figure}
		
	At the microscopic level, matter exhibits both particle and wave like properties \autoref{fig:particle-wave-duality}. This is known as the particle-wave duality and is dominant when describing slow and microscopic particles. This wave nature allows for the treatment of the particle as a wavefunction. Some of the result for having wavelike nature are the discreteness of the wavefunctions, the probabilistic behaviour of the particle, and the uncertainty of various measurable quantities of the particles. There is also tunnelling nature of particles, but this property will be overlooked in this book. 
	
	A wavefunction is a mathematical construct describing its state and the properties of the system in a wavelike fashion. From the wavefunction, by the use of mathematical operators, abstract spaces/groups with non visualizable dimensions, statistics, probability, complex planes and other math techniques, we can create, interpret and derive meaningful results. This is essentially what quantum mechanics is in simple and practical terms. Quantum mechanics is actually much deeper, but the math derivation are quite rigorous so we focus on the concepts that are deemed necessary to interpret spectroscopy. It is important to note that wavefunctions are not actually physical and are just mathematical constructs that exist in the complex plane. If all of this is confusing, do not worry as this information is just to provide directions for those who are particularly interested in the actual complexity of quantum mechanics.
		
	From the wavefunction, we can obtain various information describing the behaviour of the particle, such as its energy, location, momentum, angular momentum, and so forth. In fact, the exact form of the wavefunctions must satisfy the Hamiltonian with the associated boundary conditions of the problem.
	
	\begin{eqnarray}
		\label{eq:hamiltonian of wavefunction in cartesian}
		\hat{H} (x, y, z) \psi(x, y, z)
			&=&\left( \hat{T}(x, y, z) + \hat{V}(x, y, z) \right) \psi(x, y, z)\\
			&=&E\psi(x, y, z)
	\end{eqnarray}
	
	\begin{figure} [!ht]
		\centering
		\large
		\def\svgwidth{\columnwidth}
		\resizebox{15cm}{!}{\imginput{images/photon-electron-transition.pdf_tex}}
		\caption{{\bfseries (A)} The electron in the 1s state is absorbing an electron and transitioning into the 2p state which is higher in energy. 
			\newline
			{\bfseries (B)} The 2p electron lowers to the ground state 1s electron while also emitting a photon. \newline
			{\bfseries ***} If we assume the electrons are the same and the emission event occured right after the absorption event,  the emitted photon is randomly scattered in any direction}
		\label{fig:photon-electron-transition}
	\end{figure}	
		
	\noindent
	The Hamiltonian operator, $\hat{H}(x,y,z)$, is the energy operator and can be thought of acting on a wavefunction, $\psi(x, y , z)$, to obtain the energy of the wavefunction/particle as an eigenvalue. The eigenvalue is the constant pulled out in front of the wavefunction after the operator has operated on the wavefunction. From classical mechanics, we know that there are two types of energy, kinetic and potential energy. The Hamiltonian operator, analogous to the classical interpretation, also consist of two operators itself; the kinetic energy operator ($\hat{T}$) and potential energy operator ($\hat{V}$). 
	
	Before going any further, no wavefunctions equation will actually be shown, but instead will be represented in their bra and ket form for simplicity. $\ket{\psi_k}$, $\bra{\psi_i}$ and $\braket{\psi_i|\psi_k}$ or $\ket{k}$, $\bra{i}$ and $\braket{i|k}$. I will also be dropping the coordinates as the coordinate system used to solve the problem will depend on the symmetry of the Hamiltonian (more specifically, the symmetry of the potential energy operator $\hat{V}$). Higher importance will be shifted on to the operators, the wavefunction and their eigenvalues.
	
	\begin{eqnarray}
		\label{eq:hamiltonian of wavefunction in braket notation}
		\hat{H}(x, y, z)\ket{\psi(x, y, z)}
		&=&\left( \hat{T} + \hat{V}\right) \ket{\psi_i}\\
		&=&E_i\ket{\psi_i}
	\end{eqnarray}
	
	When a particle is confined whether by bonding, high energy barriers, or any other physical constraints, the wavefunction of the particles will take on discrete forms. This discreteness means that the wavefunction either looks like a cat, a dog, or an elephant, not somehing in between like a chimera. This confinement on the particle/wavefunction is represented by the potential operator. This discreteness originates from the applied boundary conditions on the Hamiltonian function, or partial differential equation. These discrete forms are better known as {\bfseries states} of the wavefunction and are more generally referred to as quantum numbers. Examples of systems with discrete states are the electrons bonded to a positive nucleus resulting in the n=1,2,3,4 ... $\infty$ electronic energy levels of the electrons, discrete vibrational and rotational states of molecules, and allowed resonating frequencies of photons of an optical cavity \autoref{sec:Optical Cavity and Resonance Properties}.
	\todo{make diagram of rovibrational energy levels}\
	
	Transitions from one state to \textit{can} be caused from absorption or emission of photons. With absorption of a photon, the particles increases in energy because of the transition from a lower energy state into a high energy state. For emission processes, the molecule lowers in energy to a lower energy state while emitting a photon. The process are essentially a forward and reverse pair. These transitions between the wavefunctions are also discrete, meaning only photons of specific energy may be absorbed. The energy difference for the transition is equal to the energy of the photon that is absorbed or emitted. The energy of a photon is determined by its frequency or wavelength where higher frequencies and lower wavelengths correspond to photons with higher energy. This energy relation is given by \autoref{eq:photon energy forms}. The last form is in terms of wavenumbers, unfortunately, we will be working with this alot.
	
	\begin{equation}
		\label{eq:photon energy forms}
		\begin{array}{cccccccc}
		E_{photon}=h\nu=\hbar \omega &\qquad &E_{ik}=E_{k}-E_{i}  & \qquad & E_{ik} = E_{photon}& ||| &\tilde{E}=h\tilde{\nu}
		\end{array}
	\end{equation} 
	
	Some quantum numbers have a finite number of states while others are infinite in numbers of states. Some finite number quantum numbers are the electron or nuclear spin states. Examples of infinitely numbered quantum numbers are the vibrational, rotational and electron electronic states of the matter. Mathematically, this means that these wavefunctions(solution), for a given Hamiltonian(partial differential equation), form an infinite dimensional orthogonal basis set in their Hilbert space. Do not fret as it is normal to have an infinite number of states as long as the occupation probability of the higher level states drops to zero as the energy of the states increase. 
	
	If there are an infinite number of states, then there should be an infinite number of transitions. Thankfully, there are actually a finite number of \textbf{observable} transitions because of the nature of how we are detecting these transitions. If we were to detect all transitions, interpreting spectrums of molecules would be a nightmare. Mathematically, this finiteness is because of the symmetry of the wavefunctions, electromagnetic radiation (perturbation), statistical distribution of the states, and selection rules. 
	
	An example of a \textbf{simulated} statistical distribution of states is shown in \autoref{fig:Methyl-Radical-Angular-Momentum-Distribution} describing the thermal equilibrium population of rotational states for methyl radical. As the total and component angular momentum of the states increases, the population of these states decrease to zero or are just not allowed. The transitions involving these high angular momentum states are less observable since there are few radicals populating these states to begin with. There is also the consideration of energy degenerate states. The relation between energy, probability and statistical distribution of states will be explained in \autoref{subsec:Statistical Distribution}. 
	
	\begin{figure} [!ht]
		\centering
		\Large
		\def\svgwidth{\columnwidth}
		\resizebox{14cm}{!}{\imginput{images/Methyl-Radical-Angular-Momentum-Distribution.pdf_tex}}
		\caption{blach 2 quantum numbers, c3v point group etc etc.}
		\label{fig:Methyl-Radical-Angular-Momentum-Distribution}
	\end{figure}
	
	\noindent

	
	make energy diagram showing the rotational, vibrational and maybe electronic energy levels
%-------------------------------------------------------------------------------%
%-------------------------------------------------------------------------------%
	\section{Infrared and Microwave Spectroscopy}
		\label{sec:Infrared and Microwave Spectroscopy}
		Till this point, the distinction between \underline{\textbf{rovibrational}} and \textbf{rotational} spectroscopy has been made, but not explained. The two are definitely different types of spectroscopy, but they are also connected. This is because \underline{\textbf{rovibrational}} spectroscopy is actually spectroscopy of \underline{vibrational} states coupled with \textbf{rotational} states. This coupling is due to the two different quantum number of the molecules both existing simultaneously. For example, for a given vibrational state, there exist another number of other possible rotational state. A simple analogy, is to consider a 6 sided dice and a deck of cards. The dice has 6 possible outcomes/elements while the deck of card has consists of 52 (excluding the joker) unique cards/elements. If you roll a 1 with the dice, you have 52 different cards to choose from. You can roll a 1 with a Ace of Spades, a 1 with a 5 of Hearts or a 1 with the Queen of Clubs. Of course you can also roll a 2, 3, 4, 5, or 6 while drawing any of the 52 cards. This concept can be expanded on to vibrational states being represented by the dice and rotational states being represented by the deck of cards. Hopefully you can see that there exist an infinite combinations of states for the vibrational and rotational quantum numbers. This natural spread of distributions of states between vibrational and rotation states causes rovibrational spectroscopy to be exponentially more complex than rotational spectroscopy. Rovibrational spectroscopy becomes even more complex when energy coupling of vibrational and rotational numbers are taken into account. This coupling from the nonlinear terms in the Hamiltonian will be added very late in the books lifetime.
		
		image of methyl radical rotating and vibrating here when done calculating vibrational modes\todo{asdf}.
		
		In the case of methyl radical, there are actually two quantum numbers associated with angular momentum as seen in \autoref{fig:Methyl-Radical-Angular-Momentum-Distribution}. These two quantum numbers are labelled the total and component angular momentum which we will go over shortly. The odd shape of the distribution is a result of the component angular momentum of methyl radical having a finite number of states/dimensions while the total angular momentum being infinitely dimensional. There is a 3rd quantum number M which is the projection of J onto the z axis, but I am ju
		
		These various combination of states also plays an important factor in resolution and its consequences on resolution will be explained in \autoref{sec:Factors in Resolution}. Just for knowledge purposes, electronic spectroscopy include coupling and wider distribution of combinations of \textbf{electronic} states since now we are working with electronic states, vibrational and rotational states. This makes electronic spectroscopy to be exponentially more complicated than \textbf{rovibrational} spectroscopy. On the bright side though, we can easily detect the lasing sources for electronic spectrosocopy as it is in visible region.
		
%-------------------------------------------------------------------------------%
%-------------------------------------------------------------------------------%
	\section{Rovibrational Spectrocopy}
		\label{sec:Rovibrational Spectrocopy}
		Hmm I really need help with this section Dr. Ye.
		
		talk about gas phase
		
		include rovibrational energy transitions spectrums etc diagrams.
		
		When developing rovibrational spectroscopy, it is common to use a simple diatomic as a model before moving onto more complicated systems. For efficiency however, we will jump right in to developing the principles of rovibrational spectroscopy using methyl radical model. We are starting with methyl radical since it belongs to a less symmetric point group and has more modes of vibrations. I will not provide every exact details in this book as  typing everything in latex would be a ginormous pain in the glamorous maximus. I will probably provide the complete derivation available as a Onenote book. It also would be quite redundant to go through the derivation for a diatomic molecule, then again for a slightly more complicated polyatomic. The principle theory for quantifying the spectral properties of diatomics is already inherently contained within any polyatomic system with added complexity from the increased number of degrees of freedom. This increased complexity is greatly simplfied with group theory. For teaching purposes, this allows us to treat group theory on equal footing for both perturbation theory and selection rule.
		
		One of the caveats for molecules of lower symmetry point groups (subgroup) is the increased sets of quantum numbers or modes required to describe the properties and state of the system. As shown in \autoref{fig:Methyl-Radical-Angular-Momentum-Distribution}, a 3d statistical distribution is required to represent the angular momentum for methyl as there are two quantum numbers for describing its angular momentum. A simple linear diatomic would have a 1d plot describing its distribution rotational state. This is classically equivalent to saying two degrees of freedom are required to describe the rotation of the system. With diatomics, there is only 1 quantum number. For larger complicated molecules, there are at most 3 quantum numbers for rotational spectroscopy just like how 3 degree of freedom belong to rotation for large bodied systems. The rest of the degrees of freedom go into describing the vibrational modes. The last part should be familiar to those who have studied classical mechanics. This connection of degrees of freedom and quantum numbers allows us to conveniently use results from classical mechanics as opposed to solving the actual wavefunction of the system. This is a magnificent achievement. The wave equation of large systems tend to be very difficult, if not impossible, to solve so recognizing the connection between classical mechanics and quantum mechanics allows us to solve and characterize otherwise impossible systems.
%-------------------------------------------------------------------------------%		
		\subsection{Rotational Spectroscopy}
			\label{subsec:Rotational Spectroscopy}
			Before we determining the rovibrational spectrum of methyl radical, we first need to calculate the rotational energy level. Assuming we do not have any prior knowledge of the bond lengths and angles within methyl radical, we can only approximate the energy using some initial input parameters for bond lengths and angles. 
			
			When constructing the rotation Hamiltonian model for methyl radical, there are a couple things that must kept in mind. {\bfseries We do not solve the Hamiltonian for the rotational wavefunction, but are still able to obtain the rotational eigen energies. One more time, we never obtain any rotational wavefunction of the system}. There is a relatively simple method to obtain the rotational energy level of any molecule that involves using results from classical mechanics whereby we change frames and choose a new axis system. This method is applicable to all molecules and relatively simple to solve in comparison to solving Hamiltonian for the wavefunction and their eigen energies by brute force.
			
			Since we are dealing with only angular momentum, ie spinning systems, it is natural to use spherical coordinates as we want the variables to be separable and in 3 dimensions. Cartesian coordinates would have \textbf{overly complicated} calculations and would result in identical results, if solvable. It is advisable to stay away from using the Cartesian coordinate system, but \textit{trying} to solve the rotational motion with Cartesian coordinates would provide practical experience as to why it is smart to use the correct coordinate system for your problem. But seriously though, you are going to have a bad time if you try to solve this problem in Cartesian coordinates. Mathematically, the logic would look like.
			
			\begin{eqnarray}
				\label{eq:setting-up-hamiltonian}
				\hat{H} (x, y , z) \ket{\psi_i(x, y , z)} 
				&\rightarrow&
				\hat{H} (r,\theta, \phi) \ket{\psi_i(r,\theta, \phi)}
				\\
				\hat{H} (r,\theta, \phi) \ket{\psi_i(r,\theta, \phi)} 
				&=& 
				E_i\ket{\psi_i(r,\theta, \phi)}
			\end{eqnarray}	
			
			The Hamiltonian operator we are interested in is rotational Hamiltonian,\autoref{eq:rotational Hamiltonian 1}
			
			\begin{equation}
				\label{eq:rotational Hamiltonian 1}
				\hat{H}_{rot} = \dfrac{\hat{L^2}}{2I}
			\end{equation}
			
			$\hat{L}$ is the total angular momentum operator which contains the azimuthal term of the kinetic energy operator term. Don't worry if that made no sense. This is just  mathematical way of saying that our solutions are of the spherical Harmonics and that our solutions are of the form l(l+1). We will not solve for the wave solution though. We only consider the kinetic energy terms due to rotation in this section since we assume that translation and rotational motion of the molecule are separable. This is always true. Look up why, more math required. 
			
			The overall process is to determine the center of mass frame, calculate the inertia tensor in this new frame, diagonalize the inertia tensor to produce a diagonal matrix and solve for the rotational energy in the Hamiltonian \autoref{eq:rotational Hamiltonian 1}. 			
			
		\subsection{Center of Mass Frame}
			\label{subsec:Center of Mass Frame}
			
			\begin{figure} [!ht]
				\centering
				\Large
				\def\svgwidth{\columnwidth}
				\resizebox{14cm}{!}{\imginput{images/center-mass-frame-transformation.pdf_tex}}
				\caption{Transformation from some initial frame to the center of mass frame.}
				\label{fig:center-mass-frame-transformation}
			\end{figure}	
			
			The first step to solving the rotational energy levels is to layout the molecule in an initial frame then proceed to transform it in to the center of mass frame. The center of mass is the point where the mass of the entire system can be localized so that the systems reactions to external forces and torques are simplified.
			
			When determining the center of mass frame, if possible, choose an initial frame that is close to where the center of mass frame is. For simple symmetric molecules, it should be easy to estimate. In the case of methyl radical, the center of mass must be between the carbon and 3 hydrogen atom. The carbon is more massive than the 3 hydrogen atom, therefore it must be closer to the carbon. A good choice is to place the initial origin at the center of the carbon atom then align the principle axis of the molecule along the z axis. Then placing one of the hydrogen atoms placed along the x or y axis to ease the calculation a bit in this step and in the inertial tensor step. \autoref{fig:center-mass-frame-transformation}. With the coordinates of the carbon and 3 hydrogen atoms now declared, we can begin calculating the center of mass using equations \autoref{eq:total mass} and  \autoref{eq:Center of Mass}.
			
			\begin{equation}
				\label{eq:total mass}
				M = \sum_i^4{m_i}
			\end{equation}
			
			\begin{equation}
				\label{eq:Center of Mass}
				R_{cm}=\dfrac{1}{M}\sum_i^4{m_ir_i} 
			\end{equation}


			After determing the center of mass in the initial frame \autoref{fig:center-mass-frame-transformation} (A), we shift the entire system by $-R_{cm}$ so that the center of mass is now the origin. In this new coordinate system, the rotational motion is one step closer to being separable.
			
%-------------------------------------------------------------------------------%
		\subsection{Angular Momentum and Velocity}
			\label{subsec:Angular Momentum and Velocity}		
			Angular velocity and angular momentum are similar in concept to velocity and momentum but, are used to describe the rotational quantities of a system as opposed to the transnational quantities. The easiest interpretation of angular velocity is to describe it as the rate of which the angle of a body changes with respect to its rotation axis. If a ball is spun (or an molecule) on your finger, then the direction which your finger points is the rotation axis. The motion of any point of the ball though is perpendicular about a plane which is parallel to the floor. Formally though, angular velocity is defined as
			
			\begin{equation}
				\label{eq:angular velocity}
				\vec{\omega} = \dfrac{\vec{r} \times \vec{v}}{|r|^2}
				= \braket{\omega_x,\omega_y,\omega_z}
			\end{equation}
			This quantity is a vector, meaning it has a magnitude and a direction. Our ball could spin about any plane or any axis within our coordinate system. More information on vectors can be found in the theoretical techniques chapter at \autoref{sec:Vectors and Dual Vectors}. The $\vec{r}$ is position vector of the point of interest relative to the origin. For our molecule, the origin is at the center of mass. The $\vec{v}$ vector describes the direction of which the ball is travelling. The ball as a whole is not moving anywhere but, a given point on the ball definitely has a velocity.
			
			\begin{figure} [!ht]
				\centering
				\def\svgwidth{\columnwidth}
				\resizebox{16cm}{!}{\imginput{images/angular-velocity-momentum.pdf_tex}}
				\caption{\textbf{(A)} The angular velocity of the ball is in all 3 direction. \\
				In \textbf{(B)} \textbf{(C)} and \textbf{(D)}, the component angular velocity of \textbf{(A)} in the x, y and z direction are shown respectively. From left to right, the rotational energy of the balls increase. \\
				\textbf{(D)} The direction of the angular velocity vector, $\omega_z$ are in the z direction. The spinning motion is about the x and y plane.
					}
				\label{fig:angular-velocity-momentum}
			\end{figure}	
					
			Angular momentum is a much more abstract quantity concept in comparison to angular velocity and translation momentum. Although it is the rotational analog of translation momentum, $mv$, angular momentum is not usually in the same direction as the angular velocity. The best definition that could be made is that is the product of the momentum of inertia and angular velocity of the object. (I really hate that I can't come up with a better definition)
			
			\begin{equation}
				\label{eq:angular momentum}
				\vec{L}=\textbf{I}\vec{\omega}
				=\braket{\text{I}_{x}^x\omega_x,\text{I}_{y}^y\omega_y,\text{I}_{z}^z\omega_z}=\braket{L_x,L_y,l_z}
			\end{equation}
			
			$\textbf{I}$ is the inertia tensor of a body which is described in the next in \autoref{subsec:Inertia Tensor}. Angular momentum is a very important quantity despite its poor definition since rotational energy is dependent the on angular momentum. It is also highly important due to be a conserved and observable quantity in quantum mechanics. The relationship between rotational energy and the rotational quantities are shown as
			
			\begin{equation}
				\label{eq:rotational energy classical}
				E_{\text{rot}} = \dfrac{|\vec{L}|^2}{2\textbf{I}} = \dfrac{\vec{L}^T \vec{L}}{2\textbf{I}} = \dfrac{\vec{\omega}^T\textbf{I}\vec{\omega}}{2}
			\end{equation}
			not to sure about the notation in this equation 
			
			\noindent
			Energy is a scalar quantity and thus has no direction but the quantities that define it are. From \autoref{eq:rotational energy classical}, we can take that the rotational energy of a body is dependent on the speed at which the body is rotating. The direction components are removed by taking the inner product of the angular momentum and angular velocity vector and dual vector. A dual vector (, in this case, is just the column vector or transpose of the row vector. \autoref{sec:Vectors and Dual Vectors}. In the next 2 sections, \autoref{subsec:Inertia Tensor} and \autoref{subsec:Vibrational Quantum Number}, we will see that rotational energy is also dependent on the shape, axis of rotation, and frame that we choose to work with. While we are at it, the principal axis of a molecule has two different meanings depending on which context is being discussed. In terms of inertia, the principal axis is the rotation axis that results in the greatest angular momentum, $\vec{L}$, for a given angular velocity, $\vec{\omega}$. The definition for principal axis that we will be using through the book is the symmetry definition where the it is defined to be the axis about which the object has the highest symmetry with respect to rotation. This will be elaborated more in \autoref{subsec:Group Representation}.
%-------------------------------------------------------------------------------%
		\subsection{Inertia Tensor}
			\label{subsec:Inertia Tensor}
			The moment of inertia is the quantity that describes a body tendency to resist angular acceleration. It is analogous to mass being described as a body's tendency to resist linear acceleration. The inertia tensor quantity is dependent on the shape and its size just like mass \textbf{but}, it is also dependent on the choice of our coordinate system. So what is an inertia tensor? Some of you are probably asking, what the hey are tensors? It would probably be better to discuss the uses of tensors instead going into a rigorous explanation as tensors tend to be a confusing topic for those not experienced in math. Just stare at \autoref{fig:inertia-objects-angular-directions} for a bit and focus on the shape of the object, the direction of the angular velocities and resulting angular momentum. All the objects are also assumed to be spinning about their center of mass and are assumed to be in their center of mass frame. By spinning about the center of mass and setting it as the origin done in the previous section, the rotational motion is seperable and something else important (review classical mechanics).
			
			\begin{figure} [!ht]
				\centering
				\large
				\def\svgwidth{\columnwidth}
				\resizebox{14cm}{!}{\imginput{images/inertia-objects-angular-directions.pdf_tex}}
				\caption{All the angular velocity($\vec{\omega}$) points in the $\vec{z}$ direction for \textbf{(A)}, $\vec{y}$ directon for \textbf{(B)} and $\vec{x}$ direction for \textbf{(C)}.\\
				\textbf{Spheres}: The direction of the principal axis of the object is arbritary. Because of a spheres spherical symmetry (just go with it), the angular momentum and angular velocity will always be parallel.\\
				\textbf{Irregular trigonal Pyramid}: I admit, I do not actually know which direction the angular momentums point in for angular velocities in the $\hat{x}$ and $\hat{y}$ for \textbf{(B)} and \textbf{(C)}. Only in \textbf{(A)} do the angular momentum and angular velocity match and that is because they point in the same direction as the pyramids principal axis.\\
				\textbf{Random Object}: The angular momentum and angular velocity are completely different in regular cartesian coordinates and are randomly made up in the image.
				}
				\label{fig:inertia-objects-angular-directions}
			\end{figure}
				
			Tensors are mathematical objects that take the form of some matrix of size n by m that act on scalars, vectors and matrices and map these input to a new space in the form a scalar, vector or matrix. Tensors are extremely powerful linear and nonlinear(yay nonlinear optics) mathematic tools that will \textbf{not} be deeply developed by this book. In a more intuitive sense, inertia tensors transform some angular velocity vector ($\vec{\omega}$) of an object to it angular momentum vector, ($\vec{L}$).
			
			\begin{equation}
				\label{eq:angular momentu}
				\vec{L}=I\vec{\omega}
			\end{equation}		
			
			\noindent
			For simple and highly symmetric cases, the direction of both vectors are the same ($\vec{\omega} \parallel \vec{L}$), meaning they are parallel as seen for the \text{spherical object} in \autoref{fig:inertia-objects-angular-directions} and in frame \textbf{(B)} of \autoref{fig:inertia-angular-velocity-momentum}. This is mathematically caused by the diagonal property of the inertia tensor for the object's specific configuration in the frame. This is only seen in very special cases of simple systems. Having the inertia tensor matrix, \autoref{eq:diagonal Inertia tensor}, in a diagonal form vastly simplifies the math. Luckily, there is always is a way to work diagonal inertia tensors.

			\begin{equation}
				\label{eq:diagonal Inertia tensor}
				\textbf{I} =
					\begin{pmatrix}
					I_{x}^x & 0 & 0 \\
					0 & I_{y}^y & 0\\
					0 & 0 & I_{z}^z
					\end{pmatrix} 
			\end{equation}	
			
			\noindent
			By having the inertia tensor take on a diagonal form, only the $I_{xx}$ component maps the $\omega_x$ component of $\vec{\omega}$ to the $L_x$ of $\vec{L}$. 
			
			\begin{equation}
				\begin{split}
				\label{eq:diagonal inertia tensor simplification}
				\vec{L}&=I\vec{\omega} \\
				\left\langle L_x, L_y, L_z \right\rangle& = \left\langle I_{x}^x \omega_x, I_{y}^y \omega_y, I_{z}^z \omega_z \right\rangle
				\end{split}
			\end{equation}
			
			This specificity in configuration is important as the elements in the inertia tensor are dependent on the orientation relative to the x, y, and z axis of the frame. This importance in the relative configuration of the frame and object dependency on the inertia tensor can be seen in \autoref{fig:inertia-angular-velocity-momentum} where the inertia tensor in frame \textbf{(A)} has non zero off axis elements. This now leads us to the more useful and general form of the inertia tensor, \autoref{eq:general inertia tensor form}, where the off axis elements may or may not be zero.
			
			\begin{equation}
				\label{eq:general inertia tensor form}
				\textbf{I} = 
					\begin{pmatrix}
						I_{x}^x & I_{x}^y & I_{x}^z \\
						I_{y}^x & I_{y}^y & I_{y}^z \\
						I_{z}^x & I_{z}^y & I_{z}^z
					\end{pmatrix}
			\end{equation}		
			
			To save time and space, the identity of the elements in their discrete forms will be given as opposed to an actual derivation. We use discrete forms since the atoms are treated as point masses. For a real 3d continuous object, we must use the integral form which just replaces the summation with integrals and the masses with mass densities. You can look up the derivation in any introductory classical mechanics textbook or on Wikipedia. 
			
			To use the identities of the inertia tensor, the i subscript refers to the ith atom of the molecule while k is the total number of point mass/atoms. We have to add all k contribution of atoms within the system in order to obtain the elements of the inertia tensor. For methyl radical, there are a total of 4 atoms consisting of 1 carbon and 3 hydrogen atoms so therefor $k=4$. Each atom must be given a number or label. We the label central carbon atom 1 while 2-4 are given to the hydrogen atoms. We then plug in the mass and coordinates of the atoms into \autoref{eq:I_x^x} to \autoref{eq:I_z^x} and calculate the elements of \autoref{eq:general inertia tensor form}, the general inertia tensor. The diagonal components are $I_x^x$, $I_y^y$ and $I_z^z$,
			
			\begin{eqnarray}
				\label{eq:I_x^x}
					I_{x}^x &=&\sum_0^k{m_i(y'^{2} + z'^{2})_i}\\
				\label{eq:I_yy}
					I_{y}^y &=&\sum_0^k{m_i(x'^{2} + z'^{2})_i}\\
				\label{eq:I_zz}
					I_{z}^z &=&\sum_0^k{m_i(x'^{2} + y'^{2})_i}
			\end{eqnarray}
			
			\noindent
			The next 6 elements are the off axis element.
			
			\begin{eqnarray}
				\label{eq:I_x^y}
					I_x^y = I_y^x &=&-\sum_0^k{m_i(x'y')_i}\\
				\label{eq:I_y^z}
					I_{y}^{z} = I_{y}^{z} & = &-\sum_0^k{m_i(y'z')_i}\\
				\label{eq:I_z^x}
					I_{z}^{x} = I_{x}^{z} & = &-\sum_0^k{m_i(z'x')_i}
			\end{eqnarray}
			
			In comparison to \autoref{eq:diagonal inertia tensor simplification}, the components of the angular momentum $\vec{L}$ are much complicated for a non-diagonal inertial tensor. Once we are done, we take a look at the inertia tensors and our molecule to check if our calculation agree with what we can already visually and intuitively understand.
			
			The inertia tensor can sometimes be made diagonal just by re-orientating the molecule in a more symmetrical orientation relative to the axes. If reorientation is impossible, unsure or do not want to visualize anything, it is safe to perform operations on the matrix to diagonalize the inertia tensor. By mathematically diagonalizing the inertia tensor, we are defining a new set of orthogonal axes in this a new space. 	
						
			\begin{equation}
				\label{eq:Inertia tensor}
				\textbf{I} =
				\begin{pmatrix}
					I_x^x & I_x^y & I_x^z \\
					I_y^x & I_y^y & I_y^z \\
					I_z^x & I_z^y & I_z^z
				\end{pmatrix}
				\Longrightarrow
				\tilde{\textbf{I}} =
				\begin{pmatrix}
					I_{a} & 0 & 0 \\
					0 & I_{b} & 0 \\
					0 & 0 & I_{c}
				\end{pmatrix} 
			\end{equation}	
			
			The object is not transforming, we are just choosing a new basis set for the inertia tensor. This visually looks like \autoref{fig:principal-axis-object}.
			
			\begin{figure} [!ht]
				\centering
				\large
				\def\svgwidth{\columnwidth}
				\resizebox{16cm}{!}{\imginput{images/principal-axis-object.pdf_tex}}
				\caption{a}
				\label{fig:principal-axis-object}
			\end{figure}
			
			This new basis allows us to take advantage of working with an orthogonal basis set just as with working with the highly symmetric systems with the already diagonalized inertia tensors. This mathematically translate to meaning that in the new basis set, angular velocity and angular momentum, $\vec{L}$, are parallel unlike in Cartesian axes basis. This can again be seen in \autoref{fig:inertia-angular-velocity-momentum} where the angular velocity and angular momentum, $\vec{L}$, are parallel in frame \textbf{(B)}. This set of axes belonging to the object is labelled the principal axes of inertia or inertial principal axes. This is unfortunately very similiarly named to the symmetry definition of principal axis of a system. {\bfseries It is always best(ideal) to work with orthogonal basis sets as taking the dot product of elements belonging to a orthongal basis set results in exactly 0, not 0.0000000001 or -0.00000000000001, in other words, 0 is not an approximation.} 
			
			
			\begin{equation*}
				\{\hat{x}, \hat{y}, \hat{z}\} \Longrightarrow \{\hat{a}, \hat{b}, \hat{c}\}
			\end{equation*}
			
			Back to real life, most cases result in non-diagonal inertia tensors as most real life objects tend to never have perfect symmetry. In summary, if the direction of both vectors are non parallel, then the inertia tensor must be diagonalized.
			
			\begin{figure} [!ht]
				\centering
				\large
				\def\svgwidth{\columnwidth}
				\resizebox{16cm}{!}{\imginput{images/inertia-angular-velocity-momentum.pdf_tex}}
				\caption{In frame \textbf{(A)} the angular momentum and angular velocity do not point in the same direction because the inertia tensor is undiagonalize. In frame \textbf{(B)}, the inertia tensor is diagonalized.}
				\label{fig:inertia-angular-velocity-momentum}
			\end{figure}
			
			For highly symmetrical objects, there tends to be some simple configurations that result in the inertia tensor being naturally diagonalize like in frame \textbf{(B)} of \autoref{fig:inertia-angular-velocity-momentum}. Spherically symmetric objects with consistent density have diagonal inertia tensors no matter what.
			
			make assymetrical molecule with cartesian axes and principal axes
	
			Before we begin the calculation of the rotation energy of a methyl radical, we need to calculate the inertia along various axes. For most \text{molecules} (like real life objects), rotation about one of the 3 Cartesian axes would result in an angular momentum direction that is non parallel to that direction of rotation. Only for highly symmetric molecules in a specific frame do the angular momentum vector, $\vec{L}$, and angular velocity vector, $\vec{\omega}$, point in the same direction). The purpose of the inertia tensor is to provide a framework for which the angular velocity can be mapped to the corresponding angular momentum, $\vec{L}$. You can think of the inertia tensor matrix as a transformation 3x3 matrix that maps a vector to another vector in the new space. If none of this makes sense, just solve for the elements and diagonalize the matrix for $I_a$, $I_b$, and $I_c$. The convention for the ordering of $I_a$, $I_b$, and $I_c$ is
			
			\begin{equation}
				I_a \leq I_b \leq I_c
			\end{equation}
			
			\noindent
			This convention is quite bad though and the way components are ordered does not matter. The practical discussion on symmetry is on \autoref{subsec:Molecular Spinning Tops}.
			
			Now back to continuing the calculation for the rotational energy levels of methyl radical. Since methyl radical is highly symmetric, we set the principal axis of the atom to point along the z axis with one of the hydrogen atom lying on either the x or y axis, and set the center of mass to be the origin, the inertia tensor already diagonalized. We do need to perform any diagonalizing operations or calculations. If we had skipped any of these 4 steps, the inertia tensor would have non zero off axis elements. This whole section is overkill for this molecule, but is vital for calculating and understanding less symmetrical systems such as the random object or the pyramid in \autoref{fig:inertia-objects-angular-directions}, molecules with different conformational structure such as cis or trans, and essentially 99\% of molecules.
			
			So after going through the whole process and specifically solving for the diagonalize inertia tensor of methyl radical, we arrive at \autoref{eq:diagonal inertia tensor methyl radical},
			\begin{equation}
				\label{eq:diagonal inertia tensor methyl radical}
				\tilde{\textbf{I}}= 
					\begin{pmatrix}
						I_{a} & 0 & 0\\
						0 & I_{a} & 0\\
						0 & 0 & I_{c}
					\end{pmatrix}
			\end{equation}
			where $I_c$ is the principal axis of the molecule and the relative relation of the 3 moments of inertia are
			\begin{equation}
				\label{eq:inertia tensor elements of methyl radical}
					I_a = I_b, \quad I_a > I_c
			\end{equation}
			
			\noindent
			The relative values of the diagonalized inertial tensor dictate the number of quantum numbers we need to describe the rotational energy levels as well as their relative relations. The quantum numbers that we speak of will of course come the angular momentum operator. The relation between the resulting quantum numbers will be discussed in \autoref{subsec:Angular Momentum} and the resulting types of symmetries in \autoref{subsec:Molecular Spinning Tops}, after we obtain the eigen energies.
%-------------------------------------------------------------------------------%		
		
			
%-------------------------------------------------------------------------------%			
		\subsection{Rotational Energy}
			\label{subsec:Vibrational Quantum Number}
			Now with our angular momentum operators defined, we solve the rotational Hamiltonian for methyl radical.
			
			\begin{equation}
				\label{eq:rotational Hamiltonian 2}
				\hat{H}_{rot} = \dfrac{\hat{L}^2}{2\textbf{I}} 
			\end{equation}
		
			\noindent	
			Remember that we are using results from classical mechanics in the above two subsection to solve for the energy levels. We do not obtain wavefunctions at all. That was the whole point of determining the moment of inertia of methyl radical and going through the process of making sure it is diagonalized. \autoref{eq:rotational Hamiltonian 2} is always true no matter which basis set we choose to work with, but the amount of work required will vary. 
			
			As stated in the previous section, \autoref{subsec:Inertia Tensor}, working with greatly simplifies the math hence the working with the principle axes of inertia for the object. If we try to solve solve the Hamiltonian for system like the random object in \autoref{fig:principal-axis-object} using the $\hat{x}$, $\hat{y}$, and $\hat{z}$, we would have to work  with this kind of mess
			
			\begin{equation*}
				\label{eq:rotational energy carteisan}
				\begin{split}
				\hat{H}_{rot} &= \dfrac{\hat{L}^2}{2\textbf{I}} \\
				E_{rot} &=\dfrac{\vec{L}^2}{2\textbf{I}} 
				=\dfrac{\textbf{I}\omega^2}{2}
				=\dfrac{1}{2}
				\begin{pmatrix}
					\omega^x &	\omega^y & \omega^z
				\end{pmatrix}
				\begin{pmatrix}
					I_x^x & 0 & 0  \\
					0 & I_y^y & I_y^z\\
					0 & I_z^y & I_z^z
				\end{pmatrix} 						
				\begin{pmatrix}
				 \omega_x\\
				 \omega_y\\
				 \omega_z
				\end{pmatrix}\\
				& = \dfrac{1}{2} 
				\begin{pmatrix}
					\omega^x &	\omega^y & \omega^z
				\end{pmatrix}
				\begin{pmatrix}
					I_x^x\omega_x  \\
					I_y^y\omega_y + I_y^z\omega_z\\
					I_z^y\omega_y + I_z^z\omega_z
				\end{pmatrix} 			\\	
				& = \dfrac{1}{2}
				\left( 
					\left[
					\omega^xI_x^x\omega_x + \omega^yI_y^y\omega_y + 	\omega^zI_z^z\omega_z
					\right] 
					+ \omega^yI_y^z \omega_z + \omega^zI_z^y\omega_y
				\right)
				\end{split}
			\end{equation*}
			
			\noindent
			It should be clear why we will not go any further with working with the ($\hat{x}$, $\hat{y}$, and  $\hat{z}$) basis set as the math does not come out cleanly due to the non-diagonal matrix. This would result in coupling of energy between the . If we work with the principal axes of inertia ($\hat{a}$, $\hat{b}$, and  $\hat{c}$) however, then we get nice clean math in the form of. Once the theory and derivations have been laid out, they will be used to solve the final form of the rotational energy levels of methyl radical.
			
			\begin{equation*}
				\label{eq:rotational hamiltonian principal axes of inertia}
				\begin{split}
					\hat{H}_{rot} &= \dfrac{\hat{L}^2}{2\tilde{\textbf{I}}} \\
					E_{rot} &=\dfrac{\vec{L}^2}{2\tilde{\textbf{I}}}  =\dfrac{\tilde{\textbf{I}}\omega^2}{2} =
					\dfrac{1}{2}
					\begin{pmatrix}
						\omega_a & \omega_b & \omega_c
					\end{pmatrix}
					\begin{pmatrix}
							I_{a} & 0 & 0  \\
							0 & I_{b} & 0\\
							0 & 0 & I_{c}
					\end{pmatrix}
					\begin{pmatrix}
						\omega_a\\
						\omega_b\\
						\omega_c
					\end{pmatrix}\\
					&=\dfrac{L_a^2}{2I_a} + \dfrac{L_b^2}{2I_b} + \dfrac{L_c^2}{2I_c}
				\end{split}
			\end{equation*}
			
			Anyways, hopefully you readers are convinced that working with the principal axes of inertia and intuitively understand why changing to orthogonal basis is convenient. So with the final results in classical mechanics, we convert $E_{rot}$, $L_a$, $L_b$, and $L_c$ to their respective Hamiltonian operators, $\hat{H}_{rot}$, $\hat{L}_a$, $\hat{L}_b$ and $\hat{L}_c$. We then arrive at the final form of the Hamiltonian, \autoref{eq:rotational Hamiltonian with principal axes 1}, which we will work with for solving generally solving rotational energy of molecules. 
			
			\begin{eqnarray}
				\label{eq:rotational Hamiltonian with principal axes 1}
				\hat{H}_{rot}  &=& \dfrac{\hat{L}_a^2}{2I_a} + \dfrac{\hat{L}_b^2}{2I_b} + \dfrac{\hat{L}_c^2}{2I_c}\\
				\hat{H}_{rot}&=&\dfrac{\hat{L}_a^2 + \hat{L}_b^2 + \hat{L}_c^2}{I_b} + \hat{L}_a^2\left(\dfrac{1}{2I_a}-\dfrac{1}{2I_b}\right) + \hat{L}_c^2\left(\dfrac{1}{2I_c}-\dfrac{1}{2I_b}\right)
				\label{eq:rotational Hamiltonian with principal axes 2}
			\end{eqnarray}			
			
			Now applying the results from methyl radical, we know that $I_a = I_b$ and $I_a > I_c$ from \autoref{eq:inertia tensor elements of methyl radical}, and proceed to plug in these reslts into \autoref{eq:diagonal inertia tensor methyl radical}.
			
			\begin{equation}
				\label{eq:rotational Hamiltonian for methyl radical}
				\hat{H}_{rot}  =  \dfrac{\hat{L}^2 - \hat{L}^2_c}{2I_a} + \dfrac{\hat{L}_c^2}{2I_c}
			\end{equation}				
			
			The reason why 2 (theres 3 actually, i dont understand the 3rrd one) quantum numbers are required to describe the rotational motion of the molecule is because there are two diffrent operators involved in describing its motion. The total angular momentum operator, $L^2$, and $\hat{c}$ component angular momentum operator, $L^2_c$.
			
			\begin{figure} [!ht]
				\centering
				\large
				\def\svgwidth{\columnwidth}
				\resizebox{12cm}{!}{\imginput{images/Methyl-Radical-Energy-Level-Distribution.pdf_tex}}
				\caption{asdf}
				\label{fig:Methyl-Radical-Energy-Level-Distribution}
			\end{figure}
			
			Now that we have our Hamiltonian in a desirable form, how do we extract energy values if we do not know the wavefunctions? There must be a wavefunction to operate on right?. There is actually a very clever and creative method for obtaining the eigenenergies by taking advantage of commutation relations, ladder operations and orthogonality nature of Hilbert Space/solution space. This subsection has been dragged long enough so I will develops \autoref{subsubsec:Eigen Values and Ladder Operations}. This is a topic covered in both quantum chemistry and physics classes, so I am sure many of you already are away of what the eigen values of angular momentum, vibrations and spin are. Here are the identities and results for methyl radical.
			
			\begin{eqnarray}
				\label{eq:total angular momentum eigen value}
				\hat{L}^2\ket{J,K}&=& \hbar^2\left(J\left(J+1\right)\right)\ket{J,K}\\
				\label{eq:component angular momentum eigen value}
				\hat{L}^2_c\ket{J,K} &=&\hbar^2K^2\ket{J,K} \\
				\label{eq:rotational eigen energy level of methyl radical}
				\hat{H}_{rot}\ket{J,K}  &=&\left( \dfrac{\hbar^2}{2I_a} \left( J(J+1) - K^2\right)  + \dfrac{\hbar^2}{2I_c} K^2\right)\ket{J,K} \\
				\label{eq:eigen energy level of methyl radical}
				E(J,K) &=& \dfrac{\hbar^2}{2I_a} \left( J(J+1) - K^2\right)  + \dfrac{\hbar^2}{2I_c} K^2
			\end{eqnarray}
	
			From \autoref{eq:eigen energy level of methyl radical}, we can see that our rotational wavefunction is dependent on two quantum numbers, J and K. Just knowing the wavefunction is dependent on integer values of these indices is enough. We can then define our wavefunction to live in an abstract vector product space. This is called the Solution space of the methyl radical rotational Hamiltonian to be exact. Go to \autoref{sec:Solution Space} to learn a bit more about abstract vector spaces and how they relate to vector spaces. We now move on to talking about angular momentum and what J and K mean.

			\subsubsection{Eigen Values and Ladder Operations}
				\label{subsubsec:Eigen Values and Ladder Operations}
%-------------------------------------------------------------------------------%				
		\subsection{Angular Momentum Quantum Numbers}
			\label{subsec:Angular Momentum}
			We have not delved into what the physical intepretation
			Now that our eigen energies for methyl radical have taken a more visible form(\autoref{eq:eigen energy level of methyl radical}). What we did in the previous section (\autoref{subsec:Vibrational Quantum Number}) was just manipulating the operator and its eigen values. We now begin interpreting what exactly is the component angular momentum number operator $\hat{L}_c$ and its quantum number K and how they relates to the total angular momentum operator $\hat{L}_{\text{tot}}$ and its quantum number J. Just to clarify to chemist, we are working with the angular momentum of the whole molecule, not the electron. J is not the total angular momentum of the electron where  L representing the orbital angular momentum and S representing the intrinsic spin of the electron. Molecular rotation, not electron orbital rotation. If you have made it this far in, you must already be familiar with the classical treatment of total angular momentum, otherwise the inertia tensor\autoref{subsec:Inertia Tensor} would have been complete rubbish. 
			
			Because our methyl radical is asymmetric, the components of the inertia tensor components are not all equal, $I_a = I_b \neq I_c$. Therefore, if we consider the methyl radical to be spinning with a total constant angular momentum(constant J), the rotational energy, will be dependent on which axes we are spinning about
			
			\begin{equation}
				E_{\text{rot}} =  \dfrac{|\vec{J}_a|^2}{2I_a} + \dfrac{|\vec{J}_b|^2}{2I_b} +\dfrac{|\vec{J}_c|^2}{2I_b} \neq \dfrac {|\vec{J}_\text{total}|^2}{I}
			\end{equation}
			
			Total angular momentum refers to the combined angular momentum along all of its inertial principal (Cartesian) axes, $\vec{J}_{\text{tot}} = \hat{J}_a + \hat{J}_b +\hat{J}_c$. From classical mechanics, we know that every system wants to have as little energy as possible and have as high stability as possible. To demonstrate this, we compare a spinning methyl radical to a spinning top.
			
			make drawing of spinning top
			
			We can consider our methyl radical to be equivalent to a spinning top. We all know that  spinning along the principal axis would cause it 
			
			From \autoref{eq:total angular momentum eigen value}, \autoref{eq:component angular momentum eigen value}, and \autoref{eq:rotational eigen energy level of methyl radical}, we can see that the rotational energy levels are dependent  
%-------------------------------------------------------------------------------%
		\subsection{Peterbation Theory}
			\label{subsec:Peterbation Theory}
			In order to model the interaction between the molecule and electromagnetic wave interaction, we are going to use time dependent perturbation theory where the electromagnetic wave is treated as the small perturbation. Perturbation is just a small change to a system such as magically increase the charge on an electron by 1\%, applying weak electric or magnetic field on a spin system, or adding additional terms from more realistic potentials. A specific example is considered applying an 10V electric field on an electron when the electron already experiences an electric field on the order of 10e9V from the nucleus. The perturbation is represented just by adding an additional term to the Hamiltonian to the unperturbed system.
			
			\begin{equation}
				\label{eq:time dependent pertubation hamiltonian}
				\hat{H} = \hat{H}_0 + \hat{H}_1 =i\hbar \dfrac{\partial}{\partial t}
			\end{equation}
			
			Here, $\hat{H}_0$ is the unperturbed system and is just \autoref{eq:rotational Hamiltonian 2} while $\hat{H}_1$ is the energy change to the system due to a perturbation. We will treat out perturbed Hamiltonian as the energy change due to the electric field component of an electric magnetic wave interacting with our molecule as shown in \autoref{eq:electromagnetic wave pertubation}. The vectors $\vec{\mu}$ and $\vec{E}\cos{(\omega t)}$ are the dipole moment of the molecule and electric field of the wave respectively. For this section however, we are just interested in how our rotational states interact with some pertubation.
			
			\begin{equation}
				\label{eq:electromagnetic wave pertubation}
				\hat{H}_1 = \vec{\mu}\cdot \vec{E} \cos{(\omega t)}
			\end{equation}
			
			Our wavefunction, that we did not solve, will be represented in their bra or ket vector form. It is unessential to have the exact form of the wavefunction. It is enough to have their respective eigen energies(E), initial time dependent probabilities $c_{jk}$, and knowledge of their existence. The initial and final wavefunction will be generally taken to be a superposition of all of the rotation states. We will develop the general time dependent pertubation theory equations and then set the initial and final conditions to model whether transitions are allowed or disallowed which will lead us to selection rules and group theory. 
			F
			Since we are looking at the possible transitions between the rotational states, we will work with waveequations the outt product
			
			\begin{equation}
				\label{eq:rotational wavefunction braket}
				\psi(r,t)=\sum_{jk} c_{jk}(t) \ket{j,k} e^{iE_{jk}t/\hbar}
			\end{equation}
			
			With our wave equation define, we plug \autoref{eq:rotational wavefunction braket} into \autoref{eq:time dependent pertubation hamiltonian}. Which leads to 
			
			\begin{equation}
				\begin{split}
					\label{eq:time dependent pertubation theory}
					\sum_{jk} c_{jk}\hat{H}_0 &\ket{ j, k}e^{-i E_{jk} t/ \hbar} + 
					\sum_{jk} c_{jk}\hat{H}_1\ket{j, k} e^{-i E_{j k} t/ \hbar} \\
					& =
					i\hbar \sum_{jk} {c}_{jk} \dfrac{-i E_{j k}}{\hbar}\ket{j, k}e^{-i E_{jk} t/ \hbar} +
					i\hbar \sum_{jk} \left(\dfrac{d}{dt}{c}_{jk} \right)  \ket{j, k}e^{-i E_{jk} t/ \hbar}
				\end{split}
			\end{equation}
			
			\noindent
			Since $\hat{H}_0\ket{j, k} = E_{jk}\ket{j,k}$, the first summation on the left side and right both cancel out. We are also interested in how two states interact under the small perturbation, we take the inner product of \autoref{eq:time dependent pertubation theory}.
			
			\begin{equation}
				\begin{split}
					\bra{j',k'}\sum_{jk} c_{jk} \hat{H}_1 \ket{j, k} e^{-i E_{j k} t/ \hbar}
					& =
					i\hbar \bra{j',k'}\sum_{jk} \left(\dfrac{d}{dt}{c}_{jk} \right) \ket{ j, k}e^{-i E_{jk} t/ \hbar} 
					\\
					\sum_{jk} c_{jk} \braket{j',k'|\hat{H}_1|j,k} e^{-i E_{j k} t/ \hbar} & = i\hbar \left(\dfrac{d}{dt}{c}_{j'k'} \right) e^{-i E_{j'k'} t/ \hbar}
				\end{split}
			\end{equation}
			
			Since writing (typing) expectation values of two wave equation takes up large amount of paper and time, we define $H_{j'k'}^{jk} = \braket{j',k'| \hat{H}_1 |j, k}$ and some rearrangement to isolote the $\frac{d}{dt}{c}_{jk}$ term. Remember that this is our time dependent We want to isolate the $\frac{d}{dt}{c}_{jk}$ since this changing 
			
			\begin{equation}
				\label{eq:time varying wavefunction coefficient} 
				\begin{split}
					\dfrac{d}{dt}{c}_{j'k'}
					=\frac{-i}{\hbar}
					\left[
					\sum_{jk} c_{jk} H_{j'k'}^{jk} \exp{
						\left
						(\dfrac{i(E_{j' k'} - E_{j k}) t} {\hbar}
						\right)} 
					\right]
				\end{split}
			\end{equation}
			
			\autoref{eq:time varying wavefunction coefficient} is essentially a 4 dimensional matrix or tensor (need help here, worked with arrays larger then 3 dimensions). For a given perturbation, \autoref{eq:time varying wavefunction coefficient} describes the various states living in our solution space of J and K that can couple or interact with each other. What we are really interested from equation \autoref{eq:time varying wavefunction coefficient} are the non $\bra{j',k'}$ and $\ket{j,k}$ elements of the $H_{j'k'}^{jk}$ tensor. The elements that are zero, $H_{j'k'}^{jk} = \braket{j',k'| \hat{H}_1 |j, k} = 0$, correspond to not allowed or forbidden transitions between the states $\ket{j',k'}$ and $\ket{j,k}$. We basically did all this math(must resist) just to obtain our perturbed Hamiltonian tensor $H_{j'k'}^{jk}$. We solve partially solve equation \autoref{eq:time varying wavefunction coefficient} with our perturbation of the form $\hat{H}_1 = \vec{\mu}\cdot \vec{E}$ to describe the interaction of molecules with electromagnetic radiation in the next section. This method for solving for transitions is known as the semiclassical model where the molecule is treated as a quantum system and the electromagnetic is treated classical as shown in \autoref{eq:electromagnetic wave pertubation}.
			
			\begin{equation}
				\label{eq:methyl radical first order hamiltonian}
				\textbf{H} = \sum_{jkj'k'} H_{j'k'}^{jk} = \sum_{jkj'k'}\braket{j',k'| \hat{H}_1 |j, k}
			\end{equation}
			
			\noindent
			\autoref{eq:methyl radical first order hamiltonian} can be simplified and solved by setting the initial conditions of a few states. For example, we could only be interested in calculating how ground state (j=0,k=0) interacts with the next excited states with the given perturbed Hamiltonian. Later on we will come back to \autoref{eq:time varying wavefunction coefficient} to provide a detailed insight as to how the quantum system interacts with various types of perturbation such as monochromatic, broadband radiation or sudden impulses of constant electric field etc. For now, we are just concerned with whether the transitions occur or not.
			
			\begin{eqnarray*}
				H_{j'k'}^{jk} & = & \braket{j', k'| \hat{H}_1 |j, k} = 0 \quad \text{Forbidden (Not Allowed)}\\
				H_{j'k'}^{jk} & = & \braket{j', k'| \hat{H}_1 |j, k} \neq 0 \quad \text{Allowed}
			\end{eqnarray*}
			
			There are 3 ways to determine whether are not the transitions are zero. The first method, which is not recommended, is to calculate most of the value of all elements by computer or hand (please do not do this, I am just mentioning as a joke). The 2nd method is to use selection rules (\autoref{subsec:Selection Rule}). The 3rd method is to use symmetry by using their group representations(\autoref{subsec:Group Representation}).
			
			First and second order pertubation corrections are linear and quadratic
			
			induced dipole and polarizability tensor
			
			$\hat{H} = -\vec{\mu} \cdot [\vec{E}cos(\omega_R t)]_{photon}$
			
			$\braket{\psi_{final} |\hat{H} |\psi_{initial}} \neq 0$
			
			$\braket{\psi_{final} |\hat{H} |\psi_{initial}} = 0$
			
			talk about oscillators 
			
			optical power pumping
%-------------------------------------------------------------------------------%
		\subsection{Selection Rules}
			\label{subsec:Selection Rule}
			i feel like selection rules are a part of group theory.
			selection rules for rotational spectroscopy
			
%-------------------------------------------------------------------------------%
		\subsection{Group Representation}
			\label{subsec:Group Representation}
			
%-------------------------------------------------------------------------------%
		\subsection{Rotational Transitions}
			\label{subsec:Rotational Transitions}
			
%-------------------------------------------------------------------------------%
		\subsection{Statistical Distribution}
			\label{subsec:Statistical Distribution}
			
%-------------------------------------------------------------------------------%
		\subsection{Molecular Spinning Tops}	
			\label{subsec:Molecular Spinning Tops}
			There are 5 types of spinning molecules based on the value of the elements of the diagonalized inertia tensors; linear molecule, oblate symmetric top, prolate symmetry top, spherical top and asymmetric top.
			
%-------------------------------------------------------------------------------%
%-------------------------------------------------------------------------------%
%-------------------------------------------------------------------------------%
\chapter{Optics}
	This chapter discusses the various topics in optics such as the Gaussian beam, the cavity modes, the effects of nonlinear optics, frequency spectrum of optical components.
	
	add jones vectors
	
	add fourier transform in frequency domain mostly for spectroscopy and a bit in wavelength/space and kvector domain for aligning and optical element purposes. some fourier analysis convolution of course
	
	for the record, i am just going to talk about aligning
	
	I will not teach everything about aligning as it is a hands on approach learning style
	
%-------------------------------------------------------------------------------%
%-------------------------------------------------------------------------------%

	\section{Gaussian Beam}
		\label{sec:Gaussian Beam}
		A Gaussian beam is a beam of electromagnetic radiation with a Gaussian profile for the electric and magnetic field. 
		To avoid confusion, Gaussian beams are not parallel rays of radiation and the Gaussian beam model is only accurate to within the paraxial limit since the beams exist locally around the propagation axis.
		The convention is to set the z axis as the direction of propagation where the transverse coordinates, x and y, only vary along a small range about z=0 axis, the propagation axis.
		
		The focus in this section is on the $TEM_{00}$, the simplest and most critical Gaussian mode for most laser absorption spectroscopy experiments. It contains all the parameters for describing the behavior of laser beams. The properties of interest are the k vector($k_z$), electric field intensity and direction (vector) ($\vec{E_o}$), radius of curvature of the phase fronts($R(z)$), Guoy phase, beam waist/focus ($\omega(z)$) and the Gaussian distribution of the beam $\left[ \exp{\bigg[-\dfrac{\rho^2}{\omega^2(z)}\bigg]}\right]$.
%-------------------------------------------------------------------------------%
		\subsection{Gaussian Modes and Guoy Phase}
			\label{subsec:Gaussian Modes and Guoy Phase}
			
		%	\begin{figure} 
		%		\centering
		%		\includegraphics[scale=0.9]{images/GaussianMode.png}
		%		\caption{\cite{hermite} \cite{Laguerre} The $TEM_{00}$ mode along with the other higher order Hermite modes.}
		%		\label{fig:Hermite-gaussian}	
		%	\end{figure}
			
			There are many different Gaussian modes that can resonate within a cavity, but the $TEM_{00}$ mode is the simplest and lowest order Gaussian beam solution to the Helmholtz equation. 
			Higher order modes possess nodes that can be axial in the case for Hermite are denoted by $TEM_{xy}$. Analogous to the electron orbitals, the higher the number of nodes, the higher the energy of the mode and lower the stability of the mode.
			
			The complex electric field of the $TEM_{00}$ mode is 
			
			\begin{equation}
				\label {eq:Gaussian Beam}
				\vec{\textbf{E}}(\rho,z)=\vec{\textbf{E}}_\textbf{o} \frac{\omega_{0}}{\omega(z)} \exp\bigg[\dfrac{-\rho^2}{\omega^2(z)}\bigg] \exp\bigg[ik_z z - i \tan^{-1}\bigg(\frac{z}{z_0}\bigg)\bigg]\exp\bigg[ik \dfrac{\rho^2}{2R(z)}\bigg]
			\end{equation}
			
			\begin{equation}
				\label {eq:Guoy Phase}
				\text{Guoy Phase}:\quad \zeta = \exp\left(-i\tan^{-1}{\dfrac{z}{z_o}}\right)
			\end{equation}
			
			\noindent
			Higher order modes just contain \autoref{eq:Gaussian Beam} multiplied by some linear combination of the polynomial basis sets, such as Hermite and Laguerre in combination with modification to the Guoy Phase.
		
%-------------------------------------------------------------------------------%
		\subsection{Polarization}
			\label{subsec:Polarization}
			The direction of the electric field is always orthogonal to the direction of propagation(convention is along the z axis). The electric field can point in either the x, y or a combination of both. This is known as the polarization of the beam. There can also be an associated phase shift for one of the component direction resulting in elliptically or circularly polarized light.
			
			Polarization is useful since it allows for the separation of a beam based on this property without affecting its frequency (dispersion relation) or trajectory. This can be see in figure \autoref{fig:ircease-setup} where the back reflection and input beam are seperated by their orthogonal polarization at the polarized beam splitter.
			
			include some important jones vectors for cavity reflection and. I guess include my solution to the cavity problem.
%-------------------------------------------------------------------------------%
		\subsection{The K vector}
			\label{subsec:The K vector}
			The k vector is given, by eq(\autoref{eq:kvector}), contains information about the wavelength(frequency) and direction of propagation.
			
			\begin{equation} 
				\label{eq:kvector}
				\vec{k} = \langle k_x,k_y,k_z \rangle, \qquad |\vec{k}|=\dfrac{2\pi}{\lambda}
			\end{equation} 
		
			
			\begin{figure} [!ht]
				\centering
				\def\svgwidth{\columnwidth}
				\resizebox{16cm}{!}{\imginput{images/gaus-beam-prop.pdf_tex}}
				\caption{In this figure, the Gaussian Beam's properties are shown at various z: z=0 and the Rayleigh lengths to help visualize the Gaussian Beam.
					\newline
					{\bfseries (A)}The Gausian intensity distribution along transverse planes at various z values
					\newline
					{\bfseries (B)}Intensity distribution along a section in the transverse plane at the focus and  $z=z_\alpha$.
					\newline
					{\bfseries (C)}A plot of the waist size of a Gaussian beam and constant phase of the electric field to illustrate the plane wave and spherical wave limits in the curvature of a beam
					\newline
					{\bfseries (D)} Plot of the radius of curvature, flat wavefronts have infinite curvature while spherical wave have linearly increasing curvature} 
				\label{fig:gaus-beam-prop}
			\end{figure}	
			
%-------------------------------------------------------------------------------%
		\subsection{Complex Electric Field and Intensity}
			\label{subsec:Complex Electric Field and Intensity}
			Note that the equation provided in \autoref{eq:Gaussian Beam} is the complex representation of the electric field. Using complex representations greatly simplifies the math since taking derivatives of exponentials is simpler and cleaner than taking derivatives of sinusoidal functions. The imaginary component can then be taken out by multiplying \autoref{eq:Gaussian Beam} with its complex conjugate or simply ignored, depending on the desired extracted properties. To extract information about intensity and interference patters, what desired is the magnitude of eq(\autoref{eq:Gaussian Beam}) in the complex plane. 
			The real electric field is just the real component of \autoref{eq:Gaussian Beam}, ie completely ignore the imaginary component. This is mathematically shown in equation \autoref{eq:Intensity} below.
			
			\begin{equation} 
				\vec{E}(\rho,z)=Re[\vec{\textbf{E}}(\rho,z)], \qquad
				\textit{I} \propto |\vec{\textbf{E}}(\rho,z)|^2=\vec{\textbf{E}}(\rho,z)\cdot \vec{\textbf{E}}^*(\rho,z)
				\label{eq:Intensity}
			\end{equation}
			
			The Gaussian distribution property is important in physically aligning a beam that is not visible with our eyes or a scope. By detecting the intensity of a beam, the location and width can be determined. The following equations in \autoref{eq:Intensity} can also be used to solve for the interference of pattern of 2 or more beam, an important tool for Pound-Drever-Hall locking technique and Frequency Modulation spectroscopy.
			
%-------------------------------------------------------------------------------%
		\subsection{Divergence and Spread}
			\label{subsec:Divergence}
			Gaussian beams have a natural spread characterized by their divergence.
			This results in the spreading of the beam which causes the Gaussian beam to grow larger and thus diluting the intensity of the beam.
			This natural spreading also causes the size of the beam to always be infinite, where the thinnest waist size of the beam is known as the focus. 
			
			\begin{equation}
				\omega (z)=\omega_o \sqrt{1+\left(\dfrac{z}{z_o}\right)^2}
			\end{equation}
			
			is at its lowest and is usually set to $z=0$ with $\omega(z)|_{z=0}=\omega_o$.
			Mathematically, the gaussian beam is a paraboloidal wave with respect to z in the complex plane. 
			This means that the beam is mathematically zero at $z=i z_o$ but this is along the imaginary axis and thus has no real physical meaning. 
			This is an important fact used in the ABCD law for describing the transformation of the gaussian beam via thin lens and spherical mirrors.
			
%-------------------------------------------------------------------------------%
		\subsection{Wavefront and Curvature}
			\label{subsec:Wavefront and Curvature}
			Wavefront are the locations of the beam where the phase of the beam is constant, such as 0 or $\pi$. For plane waves, phase fronts occur along a transverse plane or as shown in \autoref{fig:wavefront}, along a straight line for all x values at a fixed z value. The surface produced by the wavefronts are flat, no curvature. For spherical wave, the wave fronts are spherical with the linearly increasing radius as the wave propagates. An alternative to description of the wavefronts is curvature $\kappa = 1/R$ which is just the inverse of the radius.
			The Gaussian beam is a mix of a plane wave and spherical wave making a it essentially a mix between the two. This phase front information is contained in the 
							
			\begin{figure} [!ht]
				\centering
				\def\svgwidth{\columnwidth}
				\resizebox{16cm}{!}{\imginput{images/wavefront.pdf_tex}}
				\caption{To illustrate the curvature of the wavefronts of various models} 
				\label{fig:wavefront}
			\end{figure}	
			
			The Gaussian beam can be approximated by a plane wave close to the focus or if the beam well collimated. 
			A collimated beam has a very low spread and is approximately the same diameter at large propagation distances. 	
			
%-------------------------------------------------------------------------------%
	\section{Thin Lens and Spherical Mirrors}
		\label{sec:Thin Lens and Spherical Mirrors}
		Here quick statements and properties of converging lens and mirrors are discussed so that the interaction of the gaussian beam and mirrors or lens are understood, as well as their connection to ray optics. From ray optics, the transformation of planewave and spherical wave limits upon interaction of spherical lens are derived where the ABCD law expands upon these results to describe more practical and realistic Gaussian beam.
		
		A common optical element everyone has seen is a magnifying glass which contains a singular large converging lens. 
		This lens can be used to magnify or reimage spherical point emitters from nearby objects or focus plane wave radiation from the sun to the focal point depending on how far away the lens is examined from.
		The focal point of a lens or mirror, in the plane wave limit, is the transverse plane where plane waves are focused to. 
		Input beams with small tilt angles are slight shifted up or down depending on the sign of the tilt, a very useful property where an iris can be placed to clean the Gaussian beam. 
		The general relation between the focus of the input wave and output wave is given below in \autoref{focal} where the 0 is at the location of the optical element. 
		For a spherical element, the magnitude of the focal length is half the radius of the spherical element, $f=R_{element}/2$ and is positive for converging lenses and negative for diverging lenses.
		
		\begin{equation}\label{focal}
			\frac{1}{f}=\frac{1}{z_{output}}-\frac{1}{z_{input}}
		\end{equation}

		\begin{figure} [!ht]
			\centering
			\def\svgwidth{\columnwidth}
			\resizebox{160mm}{!}{\imginput{images/lens-reimage-focus.pdf_tex}}
			\caption{
				This figure illustrates the connection between the simple ray optics properties of lenses and ABCD gaussian beam model. Lens are mostly transmitting at the wavelength of interest.
				\newline 
				a) Here the lens maps the point emiting spherical waves at $z=-z_o$ to $z=z_1$. There is a flip of the image at $z_1$ relative to $z_0$ which explains why magnifying glasses when viewed far enough, appear flipped.
				\newline
				b)For a gaussian beam at the spherical limit, a lens refocuses a beam at $-z_o$ to $z_1$
				\newline
				c)Plane waves are focused to the focal point. Where the plane is mapped to on the focal point depends on the angle of the beam relative to the lens axis.
				\newline
				d)For a Gaussian beam where the lens is located at the focus, very similarly to the plane wave limit, the beam is focused at the focal plane. Any beam coming in at a angle will be mapped above or below}
			\label{fig:lens-reimage-focus}
		\end{figure}
		
		Since the gaussian beam is a mix of plane and spherical wave model, the slightly more advanced ABCD law is required where the matrix transformation acts on the zero of the complex paraboloidal wave in the complex plane of the propagation axis, z.
		The location of the zero of the paraboloid is at $z=-iz_o$ which is a purely imaginary number thus has no physical interpretation, consistent with the fact that the size of a beam is always finite.
		Any transformation at the zero location in the complex plane though carries over to the real axis where the beam physically exists.
		Visually, the result of the transformation is that the wavefront curvature of the output beam is approximately matched to the curvature of the input surface.
		This is supported by the eikonal equation from ray optics \cite{SalehTeichs}.
		Illustrations can be shown below in \autoref{fig:Mirrors}. 
		The use of these results will be illustrated in mode matching and resonating beams in \autoref{sec:Optical Cavity and Resonance Properties} and \autoref{sec:Aligning the IR Cavity}.
		
		\begin{figure} [!ht]
			\centering
			\def\svgwidth{\columnwidth}
			\resizebox{160mm}{!}{\imginput{images/Mirrors.pdf_tex}}
			\caption{\cite{SalehTeichs} Mirrors are mostly reflecting at the wavelength of interest.
				a) Planar mirror with $R=\infty$ \quad b) Radius of the coated surface is some value $R_o$ \quad c) Here the focus of the beam is exactly at the focal point of the mirror which is half the radius of the surface.
			}
			\label{fig:Mirrors}
		\end{figure}	
		
		Mirrors have almost the exact same transformation properties of thin lenses except now the transformed beam travels in the reverse direction, ie the k vector is transformed. \autoref{fig:Mirrors} shows how mirrors reflect Gaussian beams where the dash lines represents the beam if it was transmitted like a lens.
		The transmitting and reflectivity property of mirrors and lenses are described by reflectivity coefficient. 
		Highly reflecting materials are classified as mirrors, they usually consist of metals or thin coatings. 
		Lens are specialized pieces of glass that are transparent to certain range of frequencies.
		Assuming minimal loses of energy to the material, the percentage of radiation that is reflected and transmitted is approximately 1. 
		The reflectivity coefficient of an optical element is $r$ and the reflectiveness of the lens is $\mathcal{R}=|r|^2$. The percentage of light reflected is given by R though and thus must be completely real.
		
		\begin{equation}\label{reflectivity}
			|r|^2+|t|^2=1, \quad |r|^2=\mathcal{R}, \quad |t|^2=\mathcal{T}, \quad \mathcal{R}+\mathcal{T}=1
		\end{equation}
		
		Note reflectivity and transmission coefficient, little r and little t, can be complex numbers resulting in phase shifts of the transformed wave. More info about ray optics and beam optics can be found in ref\cite{SalehTeichs} and \cite{steck} in their respective chapters.
		
%-------------------------------------------------------------------------------%
		\subsection{Double Lens System}
			\label{subsec:Double Lens System}
			A double lens system is exactly what a sounds like, it consists of a pair of two lenses that can be used to resize, recollimate, and or relocate the focus of a beam. Reshaping and moving the focus of a beam is important for efficient coupling of power into a cavity.
			
			\begin{figure} [!ht]
				\centering
				\def\svgwidth{\columnwidth}
				\resizebox{160mm}{!}{\imginput{images/double-lens-system.pdf_tex}}
				\caption{The important concept is that the focus and size of beam can be modified by two lens by varying the distance between the two lenses. For this plano convex lens system in particular, as the second lense is moved further away, the wider and closer the focus is to the lens.
				}
				\label{fig:double-lens-system}
			\end{figure}
%-------------------------------------------------------------------------------%
%-------------------------------------------------------------------------------%
	\section{Optical Cavity and Resonance Properties}
		\label{sec:Optical Cavity and Resonance Properties}
		An optical cavity is a type of resonator that uses mirrors with high reflectivity to store optical energy of specific frequencies. Cavities are typically used for reducing the bandwidth and/or collimation of an input Gaussian beams. The Fabry Perot is the simplest stable type of cavity for resonating Gaussian Beams consisting of 2 planar mirrors with resonant beam being plane waves. Optical power build up for frequencies that form a standing wave after 1 round trip within the cavity.The frequencies that satisfy this periodic condition. 
		
		\begin{equation}\label{eq:resonantfreq}
			\nu_q=\dfrac{cq}{2L_{eff}}, \qquad \tilde{\nu}_q=\dfrac{q}{2L_{eff}}
		\end{equation}
		
		The spacing between frequencies and wavelengths that satisfy this condition is know as the free spectral range.
		
		\begin{equation}\label{eq:FSR}
			\nu_{fsr}=\dfrac{c}{2L_{eff}}, \qquad \tilde{\nu}_{fsr}=\dfrac{1}{2L_{eff}}
		\end{equation}	
		
		If the resonant condition is not met, the radiation will interfere deconstructively with itself, similar to a Michelson interferometer, and exit back out through the input mirror.			
		
		\begin{figure} 
			\centering
			\def\svgwidth{\columnwidth}
			\resizebox{160mm}{!}{\imginput{images/feb-per-cav.pdf_tex}}
			\caption{Fabry Perot Cavity with two input beam, a resonant beam labelled by red and a non resonant beam labelled with green.}
			\label{fig:feb-per-cav}
		\end{figure}
		
		The intensity output spectrum of the cavity takes of a simple unrealistic monochromatic waves is
		
		\begin{equation} \label{eq:res}
			\centering
			{I_{out} (v)}=\dfrac{I_{max}}{1+\bigg(\dfrac{2\mathcal{F}}{\pi}\bigg)^2 \sin^2\bigg({\dfrac{\pi v}{v_{FSR}}}\bigg)}=\dfrac{I_{max}}{1+\left(\dfrac{2\mathcal{F}}{\pi}\right)^2 \sin^2\left({ \dfrac{2\pi}{\lambda}} L_{cavity}\right)}
		\end{equation}
		
		r and t are the transmission and reflection coefficient of the two reflecting surfaces of the mirrors in the cavity with r $\approx 1$.
		
		\begin{figure} [!ht]
			\centering
			\def\svgwidth{\columnwidth}
			\resizebox{160mm}{!}{\imginput{images/cav-res-profiles.pdf_tex}}
			\caption{\cite{steck}Cavity output at Various Finesse with FWHM shown. Higher Finesses result in lower broadness of the signal. The profiles with respect to wavelength and cavity length are identical to this plot. }
			\label{fig:cav-res-profiles}
		\end{figure}			

%-------------------------------------------------------------------------------%			
		\subsection{Input Beam and Cavity Coupling}
			include in this section about inputting a beam frequency to one of the resonating modes of a cavity.
%-------------------------------------------------------------------------------%	
%-------------------------------------------------------------------------------%				
	\section{Cavity Alignment}
		\label{sec:Cavity Alignment}
		This cavity section consists of 2 sub-subsections, detailing the properties of resonating Gaussian beams within a stable cavity, stability conditions, effects of improper alignment due to beam displacements and mode mismatching.
	
%-------------------------------------------------------------------------------%	
		\subsection {Resonance Stability}
			\label{ssec:ResonaceSability}
			A more practical cavity consists of spherical mirrors instead of planar mirrors discussed in \autoref{sec:Optical Cavity and Resonance Properties}. 
			The resonating beam of spherical mirrors is Gaussian beams instead of the simple plane wave in the Fabry Perot cavity. The focal point of resonating beams are dependent on the position and types of mirrors used. 
			
			The resulting modes are Gaussian since the mirrors apply the phase condition that the wavefronts of the resonating beam must approximately match the spherical curvature of the mirrors.
			Each Gaussian mode has an altered associated Guoy phase shift factor which requires each mode to have a slightly altered resonating condition in the Fabry Perot section for resonating plane waves in \autoref{sec:Optical Cavity and Resonance Properties}. The result is that different modes appear at different stroke shifts.
			
			\begin{figure} [!ht]
				\centering
				\def\svgwidth{\columnwidth}
				\resizebox{160mm}{!}{\imginput{images/cav-types.pdf_tex}}
				\caption{This figure just illustrates a quick way to visualize the beam that can resonate within a cavity consisting of two mirrors of various curvatures.
				}
				\label{fig:cav-types}
			\end{figure}	
			
			The modes that strongly resonate depends on the conditions set by the spherical mirrors and the alignment of the input beam.
			How to maximize the $TEM_{00}$ and minimize all other modes are discussed in alignment and mode matching \autoref{subsec:Beam Displacement and Mode Matching}.
		
%-------------------------------------------------------------------------------%		
		\subsection {Beam Displacement and Mode Matching}
			\label{subsec:Beam Displacement and Mode Matching}
			
			\begin{figure} [!ht]
				\centering
				\def\svgwidth{\columnwidth}
				\resizebox{150mm}{!}{\imginput{images/cav-1um-output.pdf_tex}}
				\caption{a)The large peaks correspond to the desire $TEM_{00}$ mode. b) the dominant Hermite modes boxed in purple. c) the dominant Laguerre modes are circled in green.
				}
				\label{fig:cav-1um-output}
			\end{figure}		
			I still need to investigate this more
			In order for a beam to resonate strongly in the $TEM_{00}$ mode, the propagation of the input beam must be aligned close to parallel to the cavity axis.
			The cavity axis is the axis where the resonating beam to obtain a strongly resonating $TEM_{00}$ Gaussian beam. Deviations from the cavity axis from displacements and mismatches result in no resonance or coupling of optical power into to the less stable higher order. In order for the cavity axis to be well defined, the mirror axis of both cavity mirrors must be aligned with each other. Overall, to obtain resonating conditions, the two mirror axes, cavity axis, and propagation axis of the beams must be well aligned.
			
			There are two categories of improper alignments issues, misalignments and mismatches. For misalignments, the two types of problems are transverse displacements and tilt angles. Both of these alignment issues will result in vertical or horizontal nodes appearing of the resonant beam. The higher order the mode, the larger the deviation from alignment. 
			Mismatches refers to the modes and focus of the input and resonating beam not matching. These two types of alignment issues result in stronger resonance of the Laguerre modes within the cavity. In order to match the input beam with the resonating beam(modematching), two lenses are used to reshape the input beam. When the input beam is well matched the presence of the Laguerre mode should decrease along with an increase in coupling of power into the cavity. \autoref{fig:cav-1um-output} is an image of a poorly resonating beam that is not mode matched well.
			
%-------------------------------------------------------------------------------%
%-------------------------------------------------------------------------------%
%-------------------------------------------------------------------------------%
\chapter{Laser Absorption Spectroscopy}
	\label{sec:Laser Absorption Spectroscopy}
	This chapter discusses the class of spectroscopic techniques that fall under Laser Absorption Spectroscopy (LAS). The ultimate goal of this project is to acheive NICE-OHMS technique and this chapter will develop an understanding of why this technique is chosen and how it functions. In order to utilize and understand this technique though, it is recommended that a feel and familiarity with Direct Absorption Spectroscopy (DAS), Cavity Enhanced Absorption Spectroscopy (CEAS), and Frequency Modulation Spectroscopy are developed first.	

%-------------------------------------------------------------------------------%
	\section{Direct Absorption Spectroscopy}
	
	\label{sec:Direct Absorption Spectroscopy}
		This is the most basic technique of LAS where a beam is directly propagated through a sample of length of L. It is only introduced to be provided as a comparision to the other 3 techniques. The the attenuation absorption of the laser beam is governed by Beer-Lamberts law.

		\begin{equation}
			\label{eq:BeerLamberts}
			\dfrac{I(t)}{I_o} = e^{(-\alpha L)}
		\end{equation}
		
		\noindent
		The minimum detection signal is \cite{NICE-OHMS}
		
		\begin{equation}
			\label{eq:DASlimit}
			(\alpha L)_{min} = \sqrt{\dfrac{2e \beta}{ \eta P_o}}
		\end{equation}
		
		This minimum is never reached though as noise is prominent without lock in techniques at high frequencies. This naturally leads to the implementation of frequency modulation spectroscopy technique. The minimum detection signal is also not enhanced in any way which leads to use of cavity to utilize CEAS. 
		
		\begin{figure} [!ht]
			\centering
			\def\svgwidth{\columnwidth}
			\resizebox{150mm}{!}{\imginput{images/dir-abs-spec.pdf_tex}}
			\label{fig:dir-abs-spec}
			\caption{An input beam that is attenuated by a sample about a resonant transition}
		\end{figure}		
	
%-------------------------------------------------------------------------------%
	\section{Cavity Enhanced Absorption Spectroscopy}
		\label{sec:Cavity Enhanced Absorption Spectroscopy}
		In cavity enhanced absorption spectroscopy (CEAS), (gaseous) molecules are typically within the resonating beam of the cavity.
		The purpose of using a cavity is to narrow the bandwidth of the tunable input beam and to create a greater beam intensity for an enhanced beam-sample interaction. The overall result is a stronger absorption signal and possibly narrower of spectral transition, if the bandwidth of the laser is larger then the cavity resonance width. The minimum detection limit for CEAS is the DAS detection limit (\autoref{eq:DASlimit}) with enhancement from the path length enhancement factor $\dfrac{2\mathcal{F}}{\pi}$.
		
		\begin{equation}
			\label{eq:CEASlimit}
			(\alpha L)_{min}=\dfrac{\pi}{2 \mathcal{F}}\sqrt{\dfrac{2e \beta}{\eta P_o}}
		\end{equation}
		
		The two possible types of interaction of the sample with the output beam are absorption and or scattering of the resonating beam. Here, we only account for Beer-Lambert law which adds an attenuation factor Beer-Lambert absorption factor exp(-$\alpha L_c$/2).
		Assuming the frequency of the locked beam is at resonance $v=nv_{FSR}=v_n$, then \eqref{eq:res} becomes 
		
		\begin{equation} 
			\label{eq:CEAS}
			\dfrac{I_{out}}{I_{in}^*}(v_n) \approx \dfrac{\mathcal{T}^2 }{({1 - \mathcal{R}})^2}  
			\left(
				1 - \dfrac{2 \alpha L_{cav}}{1-\mathcal{R}}
			\right)
		\end{equation}
		
		\begin{equation} 
			\label{eq:L_c}
			L_{eff}^{res}=\dfrac{2}{1-\mathcal{R}} L_c 
			\approx
			\dfrac{2\mathcal{F}}{\pi} L_c
			\propto \mathcal{F}L_c
		\end{equation}		
		
		\begin{figure} [!ht]
			\centering
			\def\svgwidth{\columnwidth}
			\resizebox{14.5cm}{!}{\imginput{images/CEAS-cartoon.pdf_tex}}
			\label{fig:CEAS}
			\caption{a) is a cartoon showing a ray of light bouncing back and forth in a Fabry-Perot cavity while interacting with a sample. b) Illustrates a realistic interaction a resonating beam with a sample }
		\end{figure}
		
		\noindent
		\autoref{eq:L_c} are true as long as the reflectivity of the reflecting surfaces is about $0.999 \approx 1$, ie almost completely reflecting and lossless, and that the bandwidth of the input laser is monochromatic. With a broadband source such as OPO, the effective path is half this value from equation \autoref{eq:L_c}. 
		The key feature here is equation \autoref{eq:L_c} which is the effective path length of the beam interacting with the sample, analogous to typical broadband UV-Vis absorption spectra, is dependent on the finesse and length of the cavity. 
		This is due to the nature of light reflecting back and forth at the mirrors inside the cavity. For a cavity with a finesse of $2200$, a 1m cavity has an effective absorption length of about 0.7 to 1.4km (10 from CEAS textbook). 
		This increases the interaction time of the beam with the molecules allowing great sensitivity in absorption signals required in detecting trace amount of cold molecular ensembles.
	
%-------------------------------------------------------------------------------%
	\section{Frequency Modulation Spectroscopy}
		\label{sec:Frequency Modulation Spectroscopy}
		The purpose of Frequency Modulation Spectroscopy is to bring the detection rate away from sources of noise from mechanical vibrations, such as impact and sound, and technical oscillations such as laser intensity fluctuations. These sources of noises range from a few hertz to a couple kilohertz for acoustic a couple hundred kHz for laser noise.  By creating sidebands and heterodyning those bands with the carrier beam to create radio frequencies beat signals, the detection rate can be brought to the MHz to GHz region thus eliminating noise from acoustic and laser noise. This results in an improved detection efficiency of the absorption signals. At the cost of detection in the GHz region, the minimum absorption detection limit is degraded by a factor of $\dfrac{J_0(\beta)J_1(\beta)}{2}$, with $J_0 \approx 1$ and $J_1 < 1 $. This degradation is due to conversion of power of the carrier to the sidebands. The resulting detection limit is therefore
		
		\begin{equation}
			\label{eq:FMSlimit}
			(\alpha L)_{min}=\dfrac{\sqrt{2}}{J_0(\beta)J_1(\beta)}\sqrt{\dfrac{2e\beta}{\eta P_o}}
		\end{equation}
		
		There are two types of Frequency Modulation Spectroscopy: one where frequency is directly modulated and indirectly by phase modulation of the carrier beam. The method that will be discussed is the indirect method because of its integration to the Noise-Immune Cavity-Enhanced Optical Heterodyne Molecular Spectroscopy (NICE-OHMS) technique by Jun Ye. NICE-OHMS is discussed in the following section, \autoref{sec:Noise-Immune Cavity-Enhance Optical Heterodyne Molecular Spectroscopy}.  A short explanation of modulation can be found in section \autoref{subsec:Modulation} and more formally in the introduction of the papers \cite{PDH Intro} and \cite{FMspec}.
		
		By phase modulating a carrier beam with a low modulation index, $\beta$, by the use of electro optic modulator (EOM), sidebands of frequencies $\pm \Omega_m $ away from the carrier beam are generated. The result is 3 frequencies now present in the beam.
		
		\begin{equation}
			\label{eq:sidebands}
			\tilde{E}_{phase}(t)\approx E_o [e^{i\omega_c t}   +   \dfrac{\beta}{2} e^{i(\omega_c +\Omega_m)t}  -  \dfrac{\beta}{2} e^{i(\omega_c -\Omega_m)t}]
		\end{equation}	
		
		The carrier frequency and sidebands can then be used to interact with the sample to determine the dispersion and absorption properties of the sample. The absorption coefficient is defined to be $\alpha$ of the spectral transitions and $\eta$ for the refractive index of the sample. It is then convenient to define $T_n=e^{-\delta_n -i \phi_n}$, $\delta_n=\alpha_n \dfrac{L}{2}$ and $\phi_n=\eta_n L\dfrac{\omega_c + n\Omega_m}{c}$ where $n=0,\pm1$ for the carrier and sidebands frequencies respectively. Equation \autoref{eq:sidebands}, after propagating through the sample of path length L and having been absorbed and its phase shifted becomes
		
		\begin{equation}
			\tilde{E}_{phase}(t)\approx E_o [T_o e^{i\omega_c t}   +   T_1 \dfrac{\beta}{2} e^{i(\omega_c +\Omega_m)t}  -  T_{-1} \dfrac{\beta}{2} e^{i(\omega_c -\Omega_m)t}]
		\end{equation}
		
		Here the absorption and phase shifted are accounted for within the $T_n$ coefficients of the carrier and sideband frequencies. What is detected though is the intensity of the beams which is proportional to the the magnitude of the complex electric field. Section(\autoref{subsec:Complex Electric Field and Intensity})
		
		\begin{equation}
			\label{eq:IndirectFMsignal}
			\begin{split}
				I(t) = \dfrac{c|\tilde{E}_o|^2}{8\pi} e \approx \dfrac{c|\tilde{E}_o|^2}{8\pi}[1-\Delta\delta\beta \cos{(\Omega_m t)+\Delta\phi\beta\sin{(\Omega_m t)}}]
			\end{split}
		\end{equation}
		
		For equation \autoref{eq:IndirectFMsignal}, to arrive at the approximation it assumed the modulation depth is small, the coefficient and refractive index is the same for all 3 frequencies, and that the n=1 sideband is being used to probe the spectral transition. We then define the pair of definitions: $\delta_{-1}=\delta_0=\bar{\delta}$, $\Delta\delta = \delta_1 -\bar{\delta}$ and $\phi_{-1}=\phi_0=\bar{\phi}$, $\Delta\phi = \phi_1 -\bar{\phi}$. A more detailed derivation can be found in \cite{FMspec}. Probing with the $n=-1$ band would result in a reverse in the polarity of the signal for the same spectral transition.
		
		Equation \autoref{eq:IndirectFMsignal} is the indirect heterodyne beat FM radio frequency produce by the phase modulation. Equation \autoref{eq:IndirectFMsignal} contains two 3 terms: the dc component, $\cos{(\Omega_m t)}$, and $\sin{(\Omega_m t)}$. The $\cos{(\Omega_m t)}$ contains information about the relative absorption loss of the sidebands and carrier frequency. The $\sin{(\Omega_m t)}$ contains information about the phase shift of the sidebands and carrier frequency. As discussed in the demodulation and heterodyne section \autoref{subsec:Demodulation by Heterodyne Principle}, dc component contains absorption and dispersion data of the sample.
		
%-------------------------------------------------------------------------------%			
	\section{NICE-OHMS}
		\label{sec:Noise-Immune Cavity-Enhance Optical Heterodyne Molecular Spectroscopy}
		Noise-Immune Cavity-Enhance Optical Heterodyne Molecular Spectroscopy (NICE-OHMS) is an advanced spectroscopy technique that combines cavity enhanced absorption spectroscopy (CEAS) with frequency modulation spectroscopy (FMS). By combining strong signal of CEAS and improved detection efficiency of FMS, we obtain the ultra-sensitive spectroscopy technique NICE-OHMS. The overall effect is to enhance absorption signal by enhancement of the path length and detection of signal at a rate to the MHz region which is far removed from prominent sources of noises. With use of this technique, we can easily reach the quantum limit level of noise. A more detailed comparison with various spectroscopy techniques with NICE-OHMS can be found in \cite{NICE-OHMS}.
		
		With a cavity, only resonant frequencies can be accepted. This is no problem for coupling in just 1 frequency, but with 3 different frequencies, it can be problematic, especially for high finesse cavities.  If the carrier beam is resonant with the cavity, the sidebands will typically not be resonant unless the splitting frequency is small, the resonance widths are large, or the splitting is the same as the free spectral range. Since we want the splitting frequency to be large so that we are the FM limit, the work around that is desired is to couple in the sidebands into adjacent cavity modes by setting the FM splitting frequency to be equal to the free spectral of the cavity. For a 1m cavity, this is $5\times 10^{-3} cm^{-1}$ or 150MHz.
		
		This technique is also immune to laser intensity and frequency fluctuations since any fluctuations present in the carrier will also be present in the sidebands. Since the signal is generated by heterodyning the sideband with the carrier, any fluctuation present in carrier will be present in the sidebands thus canceling each other out.

%-------------------------------------------------------------------------------%
%-------------------------------------------------------------------------------%
%-------------------------------------------------------------------------------%			
%-------------------------------------------------------------------------------%
%-------------------------------------------------------------------------------%
%-------------------------------------------------------------------------------%
\chapter{Absorption and Dispersion}
\label{chp:Absorption and Dispersion}
perform this calculation for a transition between two allowed rotational states and redo this (im just using this as guideline for when I do the actual calculation, ty Wolfgang Demtroder and Springer)
for transitions between the two states $\ket{i} \rightleftharpoons \ket{k}$

\begin{eqnarray}
N_i(E_i) = N \dfrac{g_i}{Z}
e^{\left(-E_i/kT\right)}
\\
Z=\sum_i{g_ie^{\left(-E_i/kT\right)}}
\end{eqnarray}
einstein coefficients in terms of terms of proprbabilty of transitions.
\begin{eqnarray}
\dfrac{d}{dt} \mathcal{P}_{ik}^{\mathrm{stim}} & = & B_{ik}\rho({\nu}) 
\\
\dfrac{d}{dt} \mathcal{P}_{ki}^{\mathrm{stim}} & = & B_{ki}\rho({\nu}) 
\\
\dfrac{d}{dt} \mathcal{P}_{ki}^{\mathrm{spon}} & = & A_{ki} 
\end{eqnarray}
assume thermal equilibrium and power in is equal to power out total Power in = Power out

\begin{equation}
\dfrac{N_k}{N_i}=\dfrac{g_k}{g_i} e^{-E_{ki}/kT}
\end{equation}	

\begin{equation}
\left[ B_{ik}\rho({\nu}) + A_{21} \right]N_2 
= 
\left[ B_{12}\rho({\nu}) \right] N_1
\end{equation}
solving for $\rho(\nu)$ in terms of einstein coefficients	
\begin{equation}
\displaystyle
\rho(E)=\dfrac{8\pi \nu^2}{c^3}\dfrac{h\nu}{e^{h\nu/kT}-1}
=
\dfrac{\dfrac{A_{ki}}{B_{ki}}}{\dfrac{g_i B_{ik}}{g_k B_{ki}} e^{E/kT} - 1}
\end{equation}


determine which relations are important when doing calculations. the relatiosn between the einstein coefficients provides a connection between the various different approaches in solving for interaction of radiation with states of a system.
\begin{equation}
\begin{split}
\displaystyle
P_{ik} &= I_0 \left(N_i -\dfrac{g_i}{g_k} N_k
\right) 
\sigma_{ik}(\omega)\Delta V \\
& = \dfrac{g_iN}{Z} \left( e^{-E_i/kT}- e^{-E_k/kT}
\right)
\Delta V \int{I(\omega)\sigma_{ik}(\omega)d\omega}
\end{split}
\end{equation}
should review the full stat mech without any of that free energy horse manure. I hate thermodynamics.	

The power absorbed is dependent on the intensity of the beam, the density of the molecules in the initial state, and the coefficient to represent the effectiveness of the coupling. $\sigma_{ik}=\pi r_{ik}^2$

Dispersion relation for transitions

\begin{eqnarray}
\kappa_i & = & \dfrac{N_i e^2}{2 \epsilon_0 m} \sum_k {
	\dfrac{\omega f_{ik}\gamma_{ik}}
	{\left(\omega_{ik}^2-\omega^2\right)+\gamma_{ik}^2\omega^2}
}\\
n_i & = & \sum_k { 1 + \dfrac{N_i e^2}{2 \epsilon_0 m} \dfrac{\left(\omega^2_{ik}-\omega^2 \right) f_{ik}}
	{\left(\omega_{ik}^2-\omega^2\right)+\gamma_{ik}^2\omega^2}
}
\end{eqnarray}

not to sure about how the summation fits or if it should be ther eat all. anyways, if the resonances are far enough from each other, then we can ignore the summation and focus on a single transition. The summation is just there for completeion

\begin{equation}
I(\nu)=I_0 e^{-\alpha(\omega)z}
\end{equation}

\begin{equation}
\alpha=2 |\vec{k}|\kappa
\end{equation}
The power absorption over a spectrum of frequencies. The power absorbed is proportional to the volume of the sample that it passes through, the intensity distribution dependent on the frequency (Gaussian for gaussian beam) as well as the absorption coefficient of the transition(s).
\begin{equation}
P_{total}=\iiint{\alpha(\omega)\vec{I}(\omega)\cdot d\vec{A} dz}d\omega
\end{equation}

long lifetimes results in emission of narrower  frequencies of emitted radiation - time frequency uncertainty. There is an absorption equivalent via optical pumping. Factors that affect the lifetime of a state are collision and stimulated emission.

\begin{equation}
content...
\end{equation}
%-------------------------------------------------------------------------------%
%-------------------------------------------------------------------------------%
%-------------------------------------------------------------------------------%
\section{Factors in Intensity}
\label{sec:Factors in Intensity}
Transition cross sections
population and weighting of states
relative energies
intensity of beam
saturation
%-------------------------------------------------------------------------------%
%-------------------------------------------------------------------------------%	

\section{Factors in Resolution}
\label{sec:Factors in Resolution}
In this section, the major sources of peak broadening and their characteristic types of distributions will be discussed. The important distributions that are considered are Gaussian, Lorentzian and Voight distributions. An important concept to also keep in mind about the distributions is their spread, where high spreads correspond to high uncertainty in the measurement of frequency of the transition. These high uncertainties in the frequency measurement are a result of their relatively short lifetimes of the excited states. More on the connection between spread,uncertainty and life time is discussed in the \autoref{subsec:Natural Linewidth}.
%-------------------------------------------------------------------------------%
\subsection{Coherent Excitation}
\subsection{Lifetime}
\subsection{Distributions}
\label{subsec:Distributions}
As the total and component angular momentum increases, so does the energy of the methyl radical. These higher energy states, relative to ground, result in instability of the radical meaning these states are less populated. There is also degeneracy?? multplicity?? to take into account, but this will be further developed in soso subsusection \todo{}\
%-------------------------------------------------------------------------------%
\subsection{Natural Linewidth}
\label{subsec:Natural Linewidth}
The natural linewidth of a spectral transition is the minimum uncertainty in the transition that can be obtained. This uncertainty, or noise, is inherent for any type of measurement as can be observed in white noise. Quantum mechanically though, this natural spread is due to energy(frequency) and time uncertainty principle

\begin{equation}
\label{eq:energy(frequency)timeuncertainty}
\begin{split}
&\Delta E \Delta t \geq \dfrac{\hbar}{2} \\
&\Delta \omega \Delta t \geq \dfrac{1}{2} \\
&\Delta \nu \Delta t \geq \dfrac{1}{4 \pi}
\end{split}
\end{equation}

\noindent
for which the lifetime of the excitation or lowering from one state to another affects the broadening of the spectral transition. From \autoref{eq:energy(frequency)timeuncertainty}, we can see that shorter lifetimes will result in larger uncertainty(spread) in the energy or freqeuncy of the transition since the uncertainty of both the energy(frequency) and time must be at least the respective quantity on the right. Interestingly, \autoref{eq:energy(frequency)timeuncertainty} also shows the relationship between the energy time uncertainty of quantum mechanics and the frequency time uncertainty from fourier transformations where $E=\hbar \omega =h \nu$. 

To expand on the concept of time and frequency uncertainty, continuous wave laser beams have very narrow frequency line widths since the beam exist for an "infinite" amount of time while pulse lasers have large frequency line widths since their duration are short and finite. A more classical and mathematically rigorous explanation can be found in chapter 3 of \cite{LaserSpec1}

%-------------------------------------------------------------------------------%
\subsection{Doppler Broadening}
\label{subsec:Doppler Broadening}
For a gaseous molecule at rest and only doppler broadened, the frequency of the photon for the absorbing transition is simply equal to the energy difference of the states. 
This is NOT TAKING into account the uncertainty in the energy of the states for simplicity.
Gaseous molecules are typically traveling very fast thus shifting the wavelength or frequency of the photon required for absorption for the transition
This can be explained by considering the frame of the molecule where it is at rest and by looking at the wavelength/frequency of the photon in that frame. 
This consideration of looking at the rest frame of the molecule leads to the relation.

\begin{equation}
\label{eq:frequencyInNewRestFrame}
\omega_a =\omega_{ik} \left(1+\dfrac{v_z}{c} \right)
\end{equation}

\begin{figure} [!ht]
	\centering
	\def\svgwidth{\columnwidth}
	\resizebox{150mm}{!}{\imginput{images/dop-broad.pdf_tex}}
	\label{fig:dop-broad}
	\caption{An input beam that is attenuated by a sample about a resonant transition}
\end{figure}	

Due to the velocity of the molecules in the same direction as the beam propagation axis following a Gaussian distribution and linearity in transformation between frames, the required resonant eigenfrequency of the photon absorption also follow the same Gaussian distribution. 
At thermal equilibrium, the velocity distribution of the absorbing molecules along the z (any actually) component follows the Gaussian distribution
leading to the Gaussian distribution of absorbed eigenfrequency about $\omega_{ik}$ of the transition.

\begin{equation}
\label{eq:EigenFrequencyGaussianDistribution}
n_i(\omega)d\omega = \dfrac{N_i c}{\omega_{ik}} \sqrt{\dfrac{m}{2 \pi k T }} \exp{\left[\frac{m c (\omega-\omega_{ik})^2}{\omega_{ik} 2kT}\right]} d\omega
\end{equation}

The final assumption is to then assume that the distribution of the intensity profile for Dopplerbroadening follows the same Gaussian Distribution followed by some simplification with the definition of its full width half max.

\begin{equation}
\label{eq:dopplerintensityGaussianDistribution}
I(\omega)=I_0 \exp{\left(-\dfrac{(\omega-\omega)^2}{0.36 \delta \omega_{D}^2}\right)}
\end{equation}

\begin{equation}
\label{eq:DopplerBroadeningFWHM}
FWHM=\delta \omega_D =\dfrac{\omega_0}{c} \sqrt{\dfrac{8kT\ln{2}}{m}}
\end{equation}

From \autoref{eq:dopplerintensityGaussianDistribution} and \autoref{eq:DopplerBroadeningFWHM}, Doppler broadening is minimized by thermally cooling the distribution of the molecules about the beam axis. This can be achieved most simply by cooling the same or more extremely by supersonic molecular expansion or deceleration techniques such as Zeeman and Stark. Deceleration techniques and super sonic expansion fall under the field cold molecules which is the purpose for the existence of this project.
%-------------------------------------------------------------------------------%
\subsection{Pressure Broadening}
%-------------------------------------------------------------------------------%
\subsection{Transit-Team Broadening}
broadening due to shape and phase of the beam when interacting with atoms/molecules

\subsection{Saturation and Power Broadening}
%-------------------------------------------------------------------------------%
%-------------------------------------------------------------------------------%	

%-------------------------------------------------------------------------------%
%-------------------------------------------------------------------------------%
%-------------------------------------------------------------------------------%
\chapter{Frequency Comb}
I'm reading the frequency comb book again. just making notes
i just realized a lot of people worked on this book. itadakimasu(bow).

Mode locked lasers generate ultrashort optical pulses they have a fixed phase relationship across the broad spectrum of frequencies

5 femtoseconds. is this the standard deviation of the Gaussian envelope?

mode locking is frequency domain (comb of frequency) but discussed in time domain (pulse)

locking of frequency and phase: $f_r$ \& $f_0$

the frequency resonance profile of cavity

sharp spacing: high finesse

can also express in wavenumber(begrudgingly)

radio frequency of the comb

regularly space train of optical pulses corresponds is created by combs of frequency. make a simulation.

o right, suppose to read up on how to lock a pulse laser

keyboard warriors are gay. no i do not care about what any of you think.

carrier envelope phase: its a Gaussian envelop: make the simulator have different envelopes (triangular, square half circle for the hell of it,

saturable absorber: Optical component that has lower losses at higher optical powers. Efficiency increases as ground state is depleted.

Q(F?) switching: modulate intracavity losses by varying the finesse of the cavity

oooooo so you do not couple a broadband source beam into the cavity and achieve resonance, a continuous wave laser beam is coupled into a laser resonator. The lasing medium is excited to produce to produce the broadband spectrum. Only frequencies that satisfy the cavity resonating conditions will exist. The iris is used to select for the $TEM_{00}?$ longitudinal mode (need to get use to calling them longitudinal modes as well stop calling them the s and p orbital modes $-\_- too much chemistry$. Can we make a frequency comb and feed into external cavity for nice ohms? two different cavity lengths? the external cavity FSR half the size of the frequency comb cavity, that way the side bands frequency do not have to overlap with the other modes of the frequency comb. 

Can use a frequency comb as a rule.

rp-photonics, you are the best

nonlinear polarization: the induced polarization of the medium due to the electric field of the radiation. The two propagate together as a polarization wave. O nonlinear as in proportional to the $E^2$ or intensity of the beam. learn about symmetries of solids.

polarization wave

Kerr effect: nonlinear polarization optical effect in crystals 

Kerr Lens Mode Locking: use of nonlinear refractive index (dependent on intensity, $E^2$) to self focus the beam. distorts the wavefronts of the beam. Higher intensity, corresponds to narrower focusing. (i think)

Hmmmm. I cant look up papers to read ...

-lasing longitudinal modes (Hermite and Laguerre etc)

higher net gain? is that more optical power efficiency

the gain mechanism can be active or passive(real or effective)

hmmm look up different lasing systems

The shortness of the pulses is limited by the finite lifetime of the excited state. I am assuming this means that short lifetimes correspond to short pulses. Short lifetime and pulses both have broad frequency spectrums, relative to long lifetimes and continuous wave beams.

nonlinear index of refraction of some material together with spatial effects or interference to produce higher net gain for shorter pulses

learn about the factors in pulse duration

short pulses require broad frequencies - time-frequency uncertainty relationship. continuous beam result in small frequency 

Kerr-lens-mode-locked Ti:sapphire (KLM Ti:sapphire) laser is widely used ( is this still relevant?) 

Which name do people use more commonly, Iris or aperture? (I will switch to aperture if everyone uses it)

prisms and/or dispersion-compensating mirrors alleviate group velocity dispersion of the gain crystal (basically fix the phase of varying wavelengths again due to nonlinear refractive index)

-look up optical cycle-

assume $\phi_{carrier envelope}$ is 0 between envelopes with identical pulses (periodic). the envelope function is center at the frequency of the carrier (do a simulation for fourier series expansion)

width of spectrum inversely proportional to the temporal width

comb spacing is inversely proportional to time between pulses $f_r = \dfrac{1}{T_{between pulses}}$

spectrometer that can resolve the comb lines does not have enough temporal resulting to separate the pulses. constructive interference occurs at $nf_r$ (assuming this condition is met for locked $\phi_{ce}$

for $\phi_{ce}$ evolving with time between pulses, $T=\dfrac{1}{f_r}$, the phase shift is  $\dfrac{\Delta \phi_{ce}}{2 \pi T}$. this is a result of a shift in the frequency comb. we go from $v_n = nf_r$ to $v_n=nf_r +f_0$

for n of order 10e6, $f_0$ is the comb offset is due to pulse to pulse phase shift (are we still in this order of magnitude at infared?)

$f_0 = \dfrac{f_r \Delta \phi_{ce}}{2\pi}= \dfrac{\Delta \phi_{ce}}{2\pi T}$

{\bfseries -add group velocity to techniques-}

$\Delta \phi_{ce} = \left( \dfrac{1}{v_g} - \dfrac{1}{v_p} \right) l_c \omega_c$

$f_0 = \dfrac{f_r \Delta \phi_{ce}}{2\pi}= \dfrac{\Delta \left( \dfrac{1}{v_g} - \dfrac{1}{v_p} \right) L \omega_c}{2\pi T} = f_r \dfrac{\Delta \left( \dfrac{1}{v_g} - \dfrac{1}{v_p} \right) L \omega_c}{2\pi}$

$v_n = nf_r + f_r \dfrac{\Delta \left( \dfrac{1}{v_g} - \dfrac{1}{v_p} \right) L \omega_c}{2\pi} = f_r \left[ n+\dfrac{\Delta \left( \dfrac{1}{v_g} - \dfrac{1}{v_p} \right) L \omega_c}{2\pi} \right]$

$L$ is the cavity length

I am pretty sure there is an error in figure 1-1. $f_0$ and $f_r$ are suppose to be swapped?

$f_r$ is measured by detection of the pulse train rate $\dfrac{1}{T}$

when optical spectrum spans an octave in frequency, one of the lasing mode frequencies must be a factor of 2 larger than one of the lower modes. bare minimum, the 8th must be twice as large as the first.

multi-heterodyne spectroscopy

locking method 1(self referencing)
double one of the lowest frequency and compare to highest frequency (report in cavity free spectral range and length later)
$2v_n - v_2n = 2 ( n f_r + f_0)-( 2nf_r + f_0) = f_0$
the beat frequency as a result of comparing the low and high frquency is $f_0$

locking method 2(external reference)
can use an external continuous wave laser to interfere with one of the lower mode then frequency double the continuous wave laser and compare to higher mode to produce $f_0 = f_{beat2} - 2 f_{beat1}$ $f_{beat1}=v_s - (nf_r +f_0)$ $f_{beat2} = 2v_s - (2nf_r +f_0)$. use appropriate amplifiers and the other thing for lowering gain for weighting factors

electric optic modulator (EOM) modulates the properties of an input laser beam. \autoref{sec:Modulation and Heterodyne Principle}

what kind of modulation are we applying?

selfphase modulation temporal variation of index using short optical pulses and intensity dependent index of refraction

coincident secondary time and space focci???

nanojoule-pulse energies

microstructure fibre for spectral broadening

keyboard warriors are still super gay

0 in group velocity dispersion causes the pulse to not spread temporally

comb generator

tracking oscillator: some local oscillator reference. mix with signal and filter to extract desired information.

KLM Ti:sapphire Laser

the v-2v interferometer contains many beats. use filter to obtain desired beat frequency.

carrier envelope phase coherence reflects how well we can identify the phase of the envelope depending on the phase of an earlier pulse. $\phi_{ce}$ is important for some experiments and its variation must be minimized 

interferometric autocorrelator: michelson interferometer with nonlinear crystal and filter. The moveable arm length is asymmetric and is a multiple of the cavity effective path length (cavity round trip for interms of time). Basically want $i^{th}$ pulse to interfere with $(i-1)^{th}$ 

$\displaystyle I_{ac}(T) = \int dt\left( E(t) + E(t + \tau ) \right)^4 $
https://www.rp-photonics.com/autocorrelators.html

optical frequency standard
%-------------------------------------------------------------------------------%
%-------------------------------------------------------------------------------%	
%-------------------------------------------------------------------------------%	
\chapter{Theoretical Techniques}
	This chapter is essentially an appendix of theoretical tools and concepts that should provide a deeper understanding of the included topics in laser absorption spectroscopy. Because of the broad nature of this field, a vast array of techniques are utilized to interpret and guide the various concepts. Most of these theoretical techniques are present just because of quantum mechanics. It is \textbf{non-essential} to be a complete master of everything in this field because of its sheer size so the various topics and associated techniques will are provided in table so so to provide a more directed learning. In general though, the larger the math background, the greater the returns in reading this book. It is non-essential, again, to understand the mathematical foundation for all these the topics. 
	
	havent made table yet
	\section{Vectors and Dual Vectors}
		\label{sec:Vectors and Dual Vectors}
		\subsection{Vector Notation}
		\label{sec:Vector Notation}
		\subsection{Dual Vectors}
		\label{subsec:Dual Vectors}
	\section{Coordinate System}
		\label{coordinate System}
		
	\section{Einstein Notation}
		\label{sec:Einstein Notation}
		This book uses a bit of Einstein notation here and there along with regular convention. This book sometimes uses an incomplete version of Einstein notation so as not to completely throw off undergrads. The purpose of using Einstein notation is "simplify" the way equations look relative to regular vector notation. Some of you may be groaning as to why you would need to learn this notation, and the simple answer is that Einstein notation is significantly easier to interpret, visually conveys more information and saves tons of paper. This can be seen when comparing equations with cross products of two generic vectors ($\vec{\nu} \times \vec{\omega}$)
		
		put in example of complicated equation that is much eaier to interpret in einstein notation. maybe maxwell stress tensor	
		
		In regular vector notation to Einstein notation, we removes summation, vector hats and arrows and switch to implying summation over all the indices, usage of subscripts and super script to imply the whether a quantity is a vector or a dual vector. For now we will treat vectors and dual vector as row and column vectors or bra and ket vectors. A formal description of vectors and their dual will be found in \autoref{sec:Vectors and Dual Vectors}.
		
		The first and easiest concept is the removal of all the summation over indices. This saves tons of paper and greatly simplify equations where we sum over 3 or more indices such as i, j, and k to n, m and p of both column and row vectors. In the following example, we will sum over two indices i and j over n and m. This makes A an n by m matrix. The curved and square brackets are used to make the math easy to follow along.
		
		\begin{equation}
			\begin{split}
				\label{eq:einstein i j to n m summation}
				A = \left[ \sum_{i=1}^{n} \left( \sum_{j=1}^{m} c_j^i a_i a^j \right) \right] 
				 = &
				\left[\sum_{i=1}^{n} 
					\left( 
						c_1^i a_i a^1 + c_2^i a_i a^2 +...+ c_{m-1}^i a_i a^{m-1} +c_m^i a_i a^m 
					\right) 
				\right] + \\
				 = &\left[ c_1^1 a_1 a^1 + c_2^1 a_1 a^2 +...+c_{m-1}^1 a_1 a^{m-1}+c_m^1 a_1 a^m \right] + \\
				& \left[ c_1^2 a_2 a^1 + c_2^2 a_2 a^2 +...+c_{m-1}^2 a_2 a^{m-1}+c_m^2 a_2 a^m  \right] \\
				& \qquad + \qquad ... \qquad+ \\
				& \left[ c_1^n a_n a^1 + c_2^n a_n a^2 +...+c_{m-1}^n a_n a^{m-1}+c_m^n a_n a^m  \right] \\
			\end{split}
		\end{equation}
		\autoref{eq:einstein i j to n m summation} can "easily" be simplified to 
		
		\begin{equation}
			\label{eq:einstein i j to n m no summation}
			A=c_j^i a_i a^j
		\end{equation}
		
		\noindent
		We really just removed the summations but, hopefully this example also showed another way of viewing matrices. With the specific elements of A being $A^i_j$. Anyways, it is a good idea to switch to einstein notation if you plan to go heavily into the math, otherwise regular vector notation is good enough.
	\section{Probability in Quantum Mech.}
		\label{sec:Probability in Quantum Mech.}
		I am intentionally being incomplete with the terminology to provide an intuitive feel. When I actually learn more about the mathematics, I will add the actual justifications and correct terminology.
		
		Here, we will be discussing the parallels between probability and quantum mechanics to provide a stronger understanding of quantum mechanics and its bizarre interpretations. In probability and quantum mechanics, the goal is to describe the properties of a system based on a set, collection, or space of elements or vectors. The quantum collection of states can be discretely valued or continuous just like discrete or continuous probability distributions.
		
		Image of rotational states, allowed resonating modes in a cavity, frequency spectrum of photon in a gaussian beam, rolls on a die and 
		
		Talk about all the derived properties using probability: variance, average value, commutation
		
		Some consequences of the probabilistic nature of quantum mechanical systems are the various uncertainty principle.
		\subsection{Expectation Value}
			\begin{equation}
				\braket{H}=\int{\psi^* \hat{H} \psi}dr=\sum{|c_n|^2 E_n}=\int{E(r)|\psi(r)|^2dr}
			\end{equation}
			
			The $c_n$ is the coefficient of the wavefunction. It is essentially the weight/importance of the eigenstate in the wavefunction. The magnitude of the $|c_n|^2$ is the probability of "finding" the wavefunction in that state. Likewise in probability, $\sum_i |c_i|^2=1$ and $\sum_i |P_i|^2=1$. I like to think of it as the portion of the wavefunction that is in that state.
			This is analogous to the average quantity in probability
			\begin{equation}
				\bar{A}=\sum_i {P_i A_i}=\int{p_i A_i da}
			\end{equation}
			
			The limits of the sum and integrals taken to be the complete space or set of states. The difference between integral and summations are probability density of states and probability of states. The simplest the difference is that when we integrate, we are "multiplying" across a variable with the $da$ or $dr$ term so the density must be the probability over the unit of integrate. So you can interpret a probability density for a given system or collection as $\dfrac{dP_i(r)}{dr}=p_i(r)$.
	

%-------------------------------------------------------------------------------%
%-------------------------------------------------------------------------------%
	\section{Solution Space}
		\label{sec:Solution Space}
		linear operatorssymmetric and antisymmetric
		
		Throughout the book, Hilbert space is mentioned a couple times alongside the set of  wavefunction. Hilbert space effectively is the \textbf{abstract vector space} that is linearly independent in addition to possessing an \textbf{inner product structure} such as dot product, orthogonality, angle , length and etc. A Hilbert space acts and behaves just like a 3d spatial vector space but, can have an arbitrary amount of dimensions. A conventional vector space can only go up to 3 dimensions (4 if you include time) while a Hilbert span can go up to n dimensions.
		
		\begin{figure} [!ht]
			\centering
			\def\svgwidth{\columnwidth}
			\resizebox{160mm}{!}{\imginput{images/euclidean-hillbert.pdf_tex}}
			\caption{\textbf{(A)} is the conventional vector space most people should be familiar with where the basis is the cartesian coordinate axes ($\hat{x}$, $\hat{y}$ and  $\hat{z}$).\\
				\textbf{(B)} is an abstract vector space  (Hilbert space) that is 3 dimensional. Say the 3 elements are the various wavefunctions of a Hamiltonian, such as a spin 1/2 system, then \textbf{(B)} can be thought of as a solution space. The basis is the $\ket{i}$, $\ket{j}$ and $\ket{k}$.
			}
			\label{fig:cav-typaaaes}
		\end{figure}	
		
		The labelled solution space has also been brought along where ever Hilbert space of any partial differential equation are involved. The label solution space in this book just refers to the collection of solutions that satisfy a given partial differential equation(Hamiltonian). For example, the wavefunctions satisfying the rotational Hamiltonian ($\hat{H}_{rot}$) of a quantum mechanical system for their own solution space(Hilbert space). The wavefunctions themselves form an orthogonal basis set. Meaning they \textbf{span} the entirety of solution space of the Hamiltonian. They are also assumed to be linearly independent, thus leading to their orthogonal nature. I am not a mathematician and I have not actually taken that many upper level math classes so no proofs.
		
		This vector treatment of wavefunction allows us to represent the hermitian operators as linear operators in the form of matrices. This is achieved by using the basis set of the solution space as the basis set of the hermitian operators. A similiar transformation was done on the inertia tensor(\autoref{subsec:Inertia Tensor}) where the basis set was the cartesian axes ($\hat{x}$, $\hat{y}$ and  $\hat{z}$) then represented into the more mathematically convient inertial principal axes ($\hat{a}$, $\hat{b}$ and  $\hat{c}$). The properties of the inertia tensor and operators were not changed, just their form (basis set).
		
		show some quantum mechanics of math of the showing how linear operators can be viewed as matrices
		
		transformation of basis sets
%-------------------------------------------------------------------------------%
%-------------------------------------------------------------------------------%
%-------------------------------------------------------------------------------%
%-------------------------------------------------------------------------------%	
	\section{Tensors Products}
		\label{sec:Tensor Products}
		\autoref{subsec:Group Representation},
		\autoref{sec:Infrared and Microwave Spectroscopy}
		
		The tensor product is a general mathematical operation on two vector spaces that result in a new vector space consisting of pairs of elements from the two original vector spaces. Mathematically this looks like $a \in A, b \in B:\rightarrow (a \otimes b) \in (A \otimes B)$. A tensor product is essentially a method of book keeping the pair of  outcomes(components of a tensor) from set A and B. In this new "combination of A and B" space, the elements are the possible combinations of elements of (a) and [b] from set (A) and [B] respectively. If A and B are 2x2 matrices with elements $a_{ij}$ and $b_{ij}$ where $1\leq i,j \leq 2$, then the tensor products of A and B can be mathematically carried as
		
		\begin{equation}
		\label{eq:tensor product of 2 by 2 vector spaces}
			\begin{split}
				A \otimes B &=
				\begin{bmatrix}
					a_{11} & a_{12}\\
					a_{21} & a_{22}
				\end{bmatrix}
				\otimes
				\begin{bmatrix}
					b_{11} & b_{12}\\
					b_{21} & b_{22}
				\end{bmatrix}
				=
				\begin{bmatrix}
					a_{11}	
					\begin{bmatrix}
						b_{11} & b_{12}\\
						b_{21} & b_{22}
					\end{bmatrix} & 
					a_{12}
					\begin{bmatrix}
						b_{11} & b_{12}\\
						b_{21} & b_{22}
					\end{bmatrix}\\
					a_{21} 				
					\begin{bmatrix}
						b_{11} & b_{12}\\
						b_{21} & b_{22}
					\end{bmatrix}
					& a_{22}
					\begin{bmatrix}
						b_{11} & b_{12}\\
						b_{21} & b_{22}
					\end{bmatrix}
				\end{bmatrix}\\
	A \otimes B &=
				\begin{bmatrix}
				a_{11}b_{11} & a_{11}b_{12} & a_{12}b_{11} & a_{12}b_{12}  \\
				a_{11}b_{21} & a_{11}b_{22} & a_{12}b_{21} & a_{12}b_{22}  \\
				a_{21}b_{11} & a_{21}b_{12} & a_{22}b_{11} & a_{22}b_{12} \\
				a_{21}b_{21} & a_{21}b_{22} & a_{22}b_{21} & a_{22}b_{22} 
				\end{bmatrix}
			\end{split}
		\end{equation}
		
		From \autoref{eq:tensor product of 2 by 2 vector spaces}, the elements of $A \otimes B$ are the matrix elements $a_{ij}b_{ij}$ or more generally for any sets of elements, $a \otimes b$. We could also discuss the elements in terms of vectors but, this should suffice for group representation and set of quantum number of methyl radical. Now we will explore product space of the outcomes of a 6 side die and the first draw from a deck of card example from \autoref{sec:Infrared and Microwave Spectroscopy} in order to develop the understanding of peak broadening due to possible combinations of total and component angular moment. The actual product space of the total and angular momentum is discussed in \autoref{}.
		
		\noindent
		Set A will be the possible outcomes of a 6 sided dice: 
		\begin{equation}
		\label{eq:set A dice rolls}
		A = \{a_1, a_2, a_3 , a_4, a_5, a_6\}. 
		\end{equation}
		\noindent
		
		Set B will contain the possible outcomes of the first draw: 
		
		\begin{equation}
		\label{eq:set B deck of cards}
		B = \{b_1, b_2, b_3,...,b_{51},b_{52}\}
		\end{equation}
		
		\noindent		
		The elements $a_1$ to $a_2$ and $b_1$ to $b_{52}$ do not necessarily have to represent a specific possible outcome such as $a_1$ can represent rolling a 6 and $b_{52}$ can represents  drawing a 7 of diamonds. They just have to represent the entire set of outcomes from the possible rolls of a 6 sided die and draw of a single card from a full deck. The tensor product for sets of elements is known as a cartesian product
		
		\begin{equation}
			\begin{split}
				A \otimes B & =
				\begin{Bmatrix}
					a_1 \\
					a_2 \\
					a_3 \\
					a_4 \\
					a_5 \\
					a_6 
				\end{Bmatrix}
				\otimes
				\begin{Bmatrix}
					b_1 & b_2 & b_3 & ... & b_{51} & b_{52}
				\end{Bmatrix}\\
				& =
				\begin{Bmatrix}
					(a_{1},b_{1}) & (a_{1},b_{2}) & (a_{1},b_{3}) & ... & (a_{1},b_{51}) & (a_{1},b_{52}) \\
					(a_{2},b_{1}) & (a_{2},b_{2}) & (a_{2},b_{3}) & ... & (a_{2},b_{51}) & (a_{2},b_{52}) \\
					(a_{3},b_{1}) & (a_{3},b_{2}) & (a_{3},b_{3}) & ... & (a_{3},b_{51}) & (a_{3},b_{52}) \\
					(a_{4},b_{1}) & (a_{4},b_{2}) & (a_{4},b_{3}) & ... & (a_{4},b_{51}) & (a_{4},b_{52}) \\
					(a_{5},b_{1}) & (a_{5},b_{2}) & (a_{5},b_{3}) & ... & (a_{5},b_{51}) & (a_{5},b_{52}) \\
					(a_{6},b_{2}) & (a_{6},b_{2}) & (a_{6},b_{3}) & ... & (a_{6},b_{51}) & (a_{6},b_{52}) \\
				\end{Bmatrix}
			\end{split}
		\end{equation}
		
		I admit, I am pulling crap out of my ass with this math notation but, I want to set up for the outter product of vector spaces of multiple quantum numbers(wavefunctions) as well as using group representation to simplify the selection of observable transitions for laser absorption spectroscopy. 
		
		This is essentially what the meshgrid function from the numpy package does. It takes the two input arrays and creates this meshgrid where each index contains two values corresponding to the two input arrays. 
		
		Still very new to this dual vector and product spaces. Should probably also take some more upper level math classes. Just for kicks, we could also make this a 3d matrix if we expand the set B to show the suit of the card along with the ace to king.
		
		From \autoref{fig:Methyl-Radical-Angular-Momentum-Distribution}, the possible combination of total and component angular momentum of methyl radical , we saw each outcome consisted of a pair of results.
		
		For a given roll on a 6 side die, we could draw any 52 cards of a deck. From these two items, we could draw $6\times52=312$ results
%-------------------------------------------------------------------------------%
%-------------------------------------------------------------------------------%		
	\section{Expectation Value and Variance}
		\label{Fundamentals of Probabilty and Statistics}
		Expectation value is the "sort of the average value" of a distribution but, the weight of a the value of a random variable are taken into account. The distinction between average and expectation value are small and are usually spoken to mean the same thing. To give a feel for what weight means, we will compare the average and expectation of a fair and unfair 6 sided die to provide an understanding and intuition for weight of a value. An unfair die just refers to the sides not having equally probably rolls. 
		
		The possible rolls of a fair dice forming set A, \autoref{eq:set A dice rolls} from \autoref{sec:Tensor Products}, all have equal probability of showing up, therefore the average value is
		
		\begin{equation}
			\label{average value of set A fair dice}
			\begin{split}
				\text{Average} = \dfrac{1}{n}\sum_{i=1}^n{a_i} = \dfrac{1}{6}\sum_{i=1}^6{i} 
				&= \dfrac{1}{6}\left[1+2+3+4+5+6\right]\\
				&= \dfrac{21}{6} \\
			\end{split}
		\end{equation}
		
		The calculation for the expectation value would be
		
		\begin{equation}
		\label{expectation value of set A fair dice}
		\begin{split}
		E_{fair}(A) &= \sum_{i=1}^6{a_i P(a_i)}\\
		& =  a_1 P(a_1) + a_2 P(a_2) + a_3 P(a_3) + a_4 P(a_4) + a_5 P(a_5) + a_6 P(a_6) \\
		& =  1\dfrac{1}{6} + 2\dfrac{1}{6}+ 3\dfrac{1}{6} + 4\dfrac{1}{6} + 5\dfrac{1}{6} + 6\dfrac{1}{6}\\
		& = \dfrac{21}{6} = 3.5
		\end{split}
		\end{equation}
		
		The average and the expectation values are both the same in the case of a fair die. This is because all of the probabilities, $P(a_i)=\dfrac{1}{6}$ for $1\leq i \leq 6$, of each outcome are the same, making both equations \autoref{average value of set A fair dice} and \autoref{expectation value of set A fair dice} equivalent. Since the weight or probability of each roll is the same, the average value is also the expectation value. Now we consider the case for an unfair dice that is biased to the number 6 face appearing half the time with the other 5 rolls having equal probability. This makes $P(a_6)=\dfrac{1}{2}$ and $P(a_i)=\dfrac{1}{5}$ for $1 \leq i \leq 5$
		
		\begin{equation}
			\label{expectation value of set A unfair dice}
			\begin{split}
				E_{bias}(A) &= \sum_{i=1}^6{a_i P(a_i)}\\
				& =  a_1 P(a_1) + a_2 P(a_2) + a_3 P(a_3) + a_4 P(a_4) + a_5 P(a_5) + a_6 P(a_6) \\
				& =  1\dfrac{1}{10} + 2\dfrac{1}{10}+ 3\dfrac{1}{10} + 4\dfrac{1}{10} + 5\dfrac{1}{10} + 6\dfrac{1}{2}\\
				& = \dfrac{45}{10} = 4.5
			\end{split}
		\end{equation}
		
		Because the bias dice has a greater probability of landing on the number 6 face, the expected value ($E_{bias}(A)$) is 4.5 as opposed to 3.5, the average value of the set and expectation value of the fair dice $E_{fair}(A)$. By having the number 6 being rolled more often then the other lower numbered faces, there is an increase in the expectation value. This made the expectation value more 6 like in a sense. In terms of wavefunction, the constant coefficient(analogous to probability) of the $\ket{6}$ makes the wavefunction more $\ket{6}$ like. Anyways, the whole point of describing the difference between average and expectation is just to show the importance of the concept weight and probability of a given element. This concept is important when examining the wavefunctions or the vectors in a given system. It is perfectly acceptable to interchange the words average and expectation value when talking about the location of a distribution a measurable variable. \autoref{eq:expectation value} shows the various ways to calculate the expectation value for a given set or distribution.
			
		\begin{equation}
			\label{eq:expectation value}
			E(X)=\mu=\sum_i{x_i P(X=x_i)}=\sum_i{E(X|F_i)P(B)} =\int{xf_X(x)dx}
		\end{equation}
		
		The next important topic to cover is variance of a set or distribution. Variance in essence describes the width or shape of a given set or distribution. Distributions that are wide have larger variance relative to narrower distribution. This variance or spread is a measured by comparing the relative difference of the all the values of the set to the expectation value(average value if it makes more sense). This is mathematically done by equation \autoref{eq:variance}.
		
		\begin{equation}
			\label{eq:variance}
			V(X)=\left[E(X-E(X))\right]^2 = E(X^2)-\left(E(X)\right)^2
		\end{equation}
		
		Most of you reading this book should be very familiar with the idea of a standard deviation and its variance but, for completeness we will explore the example of the grades of two classes. The two classes will have the same average but, very different variances(spread). We will define the set of grades for one of the classes to be A and the other to be B with the elements to be the students percentage.
		
		\noindent
		Set A will have the larger variance(spread): 
		\begin{equation}
		\label{eq:set A class with large variance}
		\begin{split}
			A = \{&58, 58, 65, 60, 50, 65, 61, 49, 60, 61, 61, 56, 70, 67, 68, 59, 52, \\
			& 58, 70, 62, 60, 62, 66, 59, 57, 56, 63, 54, 67, 63, 58, 67, 56, 59, \\
			& 51, 60, 57, 53, 59, 44, 64, 67, 58, 66, 65, 71, 63, 55, 58, 56\}
		\end{split}
		\end{equation}
		
		\noindent
		Set B will have the smaller variance(spread): 
		\begin{equation}
		\label{eq:set B class with smaller variance}
		\begin{split}
			B = \{&83, 55, 58, 47, 84, 19, 66, 62, 55, 75, 45, 56, 45, 90,  7, 35, 68, \\
			& 56, 50, 58, 73, 67, 52, 75, 40, 61, 41, 88, 30, 41, 61, 60, 49, 41,\\
			& 81, 92, 80, 61, 51, 86, 68, 60, 53, 62, 71, 56, 56, 87, 57, 55\}.
		\end{split}
		\end{equation}
		
		\begin{figure} [!ht]
			\centering
			\def\svgwidth{\columnwidth}
			\resizebox{14cm}{!}{\imginput{images/class-grades.pdf_tex}}
			\caption{maybe edit this to make it more normal distributed}
			\label{fig:class-grades}
		\end{figure}
		
		\noindent
		\autoref{tab:class_stats} contains a summary of the statistics of the data. It was created using the jupyter notebook labelled  'class-grade-variance'. The average for both classes are roughly the same with Variance of 33 for class A and 320 for class B. This is visually shown in \autoref{fig:class-grades} where class B is much more widely distributed than class A dispite their similiar averages.

		\begin{table} [!ht]
			\centering
			\tabinput{datatables/class_stats.tex}
			\caption{Statistic of Class A and Class B}
			\label{tab:class_stats}
		\end{table}
		
		\noindent
		I may or may not cover conditional probability. If it comes up at anytime during the calculations, I will.
		
		Discrete Conditional Probability
		\begin{equation}
			P(B_i|A)=\dfrac{P(B_i \cap A)}{P(A)}
			=\dfrac{P(B_i)P(A|B_i)}{\displaystyle \sum_j{P(B_j)P(A|B_j)}}
			=\int_{B_i\cap A}{\dfrac{f(x)}{P(A)}dx}
		\end{equation}

		Conditional Expectation
		\begin{equation}
			E(X|F)=\sum_i{x_i P(X=x_i|F)}
		\end{equation}
%-------------------------------------------------------------------------------%
%-------------------------------------------------------------------------------%
	\section{Distributions}
		\label{sec:Distribution}
		Might break this into two sections: continuous and discrete distributions
		
		A Distribution is a tool or object that describes how a random variable is spread out(distributed). There are various types of distributions that have different shapes and behaviors. Distributions can be discrete such as the various combinations of quantum numbers or continuous such a frequency spectrum or intensity distribution of a Gaussian beam. All of the types of distributions that occur in this book will be listed here along with their properties. Along each distribution, the sections for which they occur in will also be referenced.
		
		When describing distributions, at least two important quantities are required. The expected value (or more commonly known as the average) and the spread of the distribution. The average, mean, median or expected value describes the "location" of the distribution while the spread, full width half max, and variance describes how wide or narrow the distribution is. 
	
%-------------------------------------------------------------------------------%
	\subsection{Boltzmann Distribution}
	
	\subsection{Gaussian Distribution}
		\label{subsec:Gaussian Distribution}
		The Gaussian distribution is the easiest to recognize distribution and sometimes referred to as the bell curve when referring to grade distribution of a class of students. A normal distribution is just the normalized distribution of a Gaussian distribution which describes its probability density. A normalized Gaussian distribution follows the equation
		
		\begin{equation}
		\label{eq:Gaussian Distribution}
		f(x;\mu,\sigma) = \dfrac{1}{\sigma \sqrt{2\pi}} e^{-\frac{(x-\mu)^2}{2\sigma^2}}
		\end{equation}
		
		\begin{figure} [!ht]
			\centering
			\def\svgwidth{\columnwidth}
			\resizebox{15cm}{!}{\imginput{images/Gaussian-Distribution-plots.pdf_tex}}
			\label{fig:Gaussian Distribution plots}
		\end{figure}
			
%-------------------------------------------------------------------------------%
	\subsection{Lorentian Distribution}
		\label{subsec:Lorentian Distribution}
		
		\begin{figure} [!ht]
			\centering
			\def\svgwidth{\columnwidth}
			\resizebox{15cm}{!}{\imginput{images/lorentzian-distributions.pdf_tex}}
			\label{fig:lorentzian-distributions}
		\end{figure}
		
		\begin{equation}
		f(x;\mu, \Gamma)= \dfrac{\Gamma}{\pi} \dfrac{1}{(x-\mu)^2+\Gamma^2}
		\end{equation}
		FWHM is $2\Gamma$
		
%-------------------------------------------------------------------------------%
	\subsection{Delta "Distribution"}
		\label{subsec:Delta "Distribution"}
		\begin{equation}
		\delta(x-x_o)=
		\begin{cases}{}
		\infty & x=x_o\\
		0 & \text{otherwise}
		\end{cases}
		\end{equation}
%-------------------------------------------------------------------------------%
	\subsection{Exponential Distribution}
	\label{subsec:Exponential Distribution}
	\begin{equation}
	f(x;\lambda)= \lambda e^{-\lambda x}
	\end{equation}
%-------------------------------------------------------------------------------%
	\subsection{Binomial Distribution}
	\label{subsec:Binomial Distribution}
	\begin{equation}
	\begin{pmatrix}
	n \\
	j
	\end{pmatrix}
	=
	\dfrac{n!}{j!(n-j)!}
	\end{equation}
	
	\begin{equation}
	b(n,p,k) =
	\begin{pmatrix}
	n \\
	k
	\end{pmatrix}
	p^k(p-1)^{(n-k)}
	\end{equation}
	
	$\mu = np$ $\sigma =\sqrt{npq}$
	
	its continuous counterpart is the Gaussian distribution (\autoref{subsec:Gaussian Distribution})
%-------------------------------------------------------------------------------%
	\subsection{Poisson Distribution}	
	\label{subsec:Poisson Distribution}
	
%-------------------------------------------------------------------------------%
	\subsection{Geometric Distribution}	
	\label{subsec:Geometric Distribution}
	\begin{equation}
	P(p,k)=(1-p)^{k-1}p
	\end{equation}
	
	this one might not be to relevant. I think I saw it appear in stat. mech.. Not to sure
	
	its continuous counterpart is the exponential distribution (\autoref{subsec:Exponential Distribution})			
%-------------------------------------------------------------------------------%
%-------------------------------------------------------------------------------%
	\section{Probabilistic Convolutions}
		\label{subsec:convolution}
		(this section will for sure recieve a huge revision when I learn electrical engineering. its pretty incomplete.)
		
		Taking the  convolutions of many different distributions is an important concept to have understood at a visual level, at the very least. Understanding how a convolution works will help in performing on the spot judgment and adjustments to an experiment. Understanding and knowing how to perform a convolution will provide a deeper insight on how the various instruments of laser an experiment interact. For example, if the frequency spectrum of the detected laser beam is outside of the detector spectrum, then the detector will not pick up the laser beam. Laser absorption signals are also largely convolutions of all the sources of broadening along with the detector and incident radiation frequency spectrum.
		
		A convolution is in essence a mix between two different distributions. For example, if we have two sets, A and B, and we want to find their sum, we would define a new set $C = A + B$. The new set C is a new set dependent on the elements in A and B. This new distribution C would have elements $c_{(a_i,b_j)} = a_i + b_j$. Mathematically, this would look like.
		
		\begin{equation}
			\label{eq: C = A + B convolution}
			\begin{matrix}
			a \in A & A =\{a_1, a_2, a_3\} \\
			b \in B & B =\{b_1, b_2, b_3\} \\
			C = A + B & c = a + b \\
			c  \in C & C= 
				\begin{Bmatrix}
					c_1 = a_1 + b _1,& c_2 = a_1 + b_2, & c_3 = a_1 + b_3 \\
					c_4 = a_2 + b_1, & c_5 = a_2 + b_2, & c_6 = a_2 + b_3 \\
					c_7 = a_3 + b_1, & c_8 = a_3 + b_2, & c_9 = a_3 +b_ 3
				\end{Bmatrix}
			\end{matrix}
		\end{equation}
		
		This procedure is very similar to the outer product in \autoref{eq:tensor product of 2 by 2 vector spaces} but, instead of creating a pair of elements, a sum or difference is taken, depending on the type of distribution. To mathematically carry out the calculation, we consider for the set C we pull out $(c)$ and from A we pull out $(a)$. Therefore from B we must pull out $b = (c-a)$, where $b = (c-a)$ is an element from set B.
		
		\begin{equation}
			\label{eq:convolution set}
			P(C=c)=\sum_{k=1}^3{P(A=a)P
				\left(
					B=(c-a)
				\right)}
		\end{equation}
		
		The convolution discrete from \autoref{eq: C = A + B convolution} is just to provide an easily visualizable way of thinking of convolutions. This concept can also be applied to large continuous distributions as well. We can take the convolution of two Gaussian distributions to produce a new Gaussian distribution or to be creative, we can take the convolution of a Gaussian distribution with a square box distribution. Instead of working with the probabilities of the elements in the set, we ar now working densities(distribution) functions of the various distributions.
		
		\begin{equation}
			\label{eq:convolution distribution}
			f_C(c) = f_A(a) \ast f_B(b) = \int{f_A(a) f_B (c-a) da}
		\end{equation}
		
		The subscripts just refers to the specific distribution while the distribution is a function of its elements which are form a continuum. The most useful and simplest convolution for laser absorption spectroscopy would be the convolution of two Gaussian Distributions. We will be switching to x, y and z as the variables(elements) of the distributions X, Y and Z respectively for familiarity when performing calculations. It does not matter what labels we use as the letters we assign are just labels. The labels do not affect the outcome of the math (if done correctly). So now we have Z as the distribution, instead of C, that is dependent on the X and Y distribution. The X and Y are Gaussian distributed, \autoref{eq:Gaussian Distribution} with the forms
		
		\begin{eqnarray}
			\label{eq: X gaussian distribution}
			f_X(x) & = \dfrac{1}{\sigma_X \sqrt{2 \pi}} e^{\frac{- (x-\mu_x)^2}{2 \sigma_X^2}} \\
			\label{eq: Y gaussian distribution}
			f_Y(y) & = \dfrac{1}{\sigma_Y \sqrt{2 \pi}} e^{\frac{- (y-\mu_y)^2}{2 \sigma_Y^2}} \\
			\label{eq: Z convoluted distribution of X and Y}
			f_Z(z) & = f_X(x) \ast f_Y(y)
		\end{eqnarray}
		
		\noindent
		Now with our distributions defined, we can begin our calculation. Keep in mind, what we are doing is similar to an outer product \autoref{eq:tensor product of 2 by 2 vector spaces}. (it might actually be an outter product but, I have not taken the enough upper level math classes ). If you hate math with a fiery passion, just go down to the end of the section.
		
		\begin{equation}
			\label{eq:convolution of two Gaussian functions part 1}
			\begin{split}
			f_Z(z) & = f_X(x) \ast f_Y(y) = \int{f_X(x) f_B (z-x) dx}\\
			& = \dfrac{1}{(2 \pi) \sigma_X \sigma_Y} 
			\int_{-\infty}^{\infty}{
				e^{- \frac{(x-\mu_x)^2}{2 \sigma_X^2}} e^{- \frac{(z - x - \mu_y)^2}{2 \sigma_Y^2}}dx} \\
			& = \left[
					\dfrac{1}{\sqrt{2 \pi} \sqrt{\sigma_X^2 + \sigma_Y^2}} 
				\right] 
				\left(
					\dfrac{1}{\sqrt{2 \pi}
					\left[ 
						\frac{\sigma_X \sigma_Y}{\sqrt{\sigma_X^2+\sigma_Y^2}}
					\right]}
				\right)
				\int_{-\infty}^{\infty}{e^{-\frac{(x-\mu_x)^2}{2 \sigma_X^2}}} e^{- \frac{(z-x-\mu_y)^2}{2 \sigma_Y^2}}dx 
			\end{split}
		\end{equation}
		
		The next bit of math just involves algebraic manipulation of the fractions in the exponential. The goal though is to separate the convolution into two parts, an easy to solve integral with respect to x and pretty much everything else. With proper manipulation of the math, we can set the integral to be equal to 1 if we integrate over a normalized Gaussian distribution. This integral will be found within the regular bracket. Everything else will go into the square bracket.
		
		\begin{equation}
			\label{eq:convolution of two Gaussian functions part 2}
			\begin{split}
			f_Z(z) & = \left[
					\frac{e^{-\frac{\left(z-(\mu_x + \mu_y)\right)^2}{2( \sigma_X^2 + \sigma_Y^2)}}}{\sqrt{2 \pi(\sigma_X^2 + \sigma_Y^2)}}
				\right]
				\left(				
					\dfrac{1}{\sqrt{2 \pi}
					\left[ 
						\frac{\sigma_X \sigma_Y}{\sqrt{\sigma_X^2+\sigma_Y^2}}
					\right]}
					\int_{-\infty}^{\infty}\exp{\left\{-\frac{(x-\mu_o)^2}{\frac{\sigma_X^2 \sigma_Y^2}{\sigma_X^2+\sigma_Y^2}}\right\}}dx
				\right) \\
			& = 
			\left[
				\frac{1}{\sqrt{2 \pi(\sigma_X^2 + \sigma_Y^2)}}
				e^{-\frac{\left(z-(\mu_x + \mu_y)\right)^2}{2( \sigma_X^2 + \sigma_Y^2)}}
			\right]
			\end{split} 
		\end{equation}
		
		We are left with a equation that is very nice and maneagle. I purposely left a lot of math this form to show that the convolution of a Gaussian distribution is another Gaussian distribution. Looking at \autoref{eq:convolution of two Gaussian functions part 2}, the distribution Z is just another Gaussian distribution with a variance of $\sigma_Z^2 = \sigma_X^2 + \sigma_Y^2$ and average of $\mu_z = \mu_x + \mu_y$. This leaves us with the final form of the distribution Z.
		
		\begin{equation}
			f_Z(z) = \dfrac{1}{\sigma_Z \sqrt{2 \pi}} e^{-\frac{\left( z - \mu_z \right)^2}{2 \sigma_Z^2}}
		\end{equation}
		
%-------------------------------------------------------------------------------%
%-------------------------------------------------------------------------------%
	\section{Fourier Analysis}
		\label{sec:Fourier Analysis}
		Fourier analysis is the study of which functions by me represented by a complete set of harmonic function. This statement says that every function can be decomposed into an infinite number sines and cosines or complex exponential of varying frequencies.  This topic is critical (ITS A CRIT) for understanding how the frequency spectrum of a beam is modified by optical elements, analysis of periodic signals in the time domain, the characteristic of pulsed laser beam, and understanding the effects of modulation of a signal or beam. One of the most impactful usage of fourier analysis on chemistry is the Fourier Transform Infrared Spectroscopy (FTIR). These are just some of the important aspects of fourier analysis found in this book. Understanding fourier transform, fourier series, and the many other topics of fourier analysis. This section will be broken down in to the so and so sections. First topic is 
		
		\subsection{Fourier Space}
			\label{subsec:Fourier Series}
			In this subsection, we will introduce the concepts of time and frequency domain and the connection between the two. If you do not like math, designing systems or signal processing, this subsection should be enough as we will first focus on concepts. Before we can even talk about the math, theory and results of fourier transformations, it is more beneficial for such a wide audience to first develop an intuition for the duality of time and frequency domain as everyone should understand. Visually and conceptually understanding fourier transform properties allows for quick interpretation of electrical signals and optical systems and so forth.
			
			As stated at the beginning of the \autoref{sec:Fourier Analysis}, we stated that any function can be represented as a complete set of sines and cosines or complex exponential but, we will focus our definition of function to only include distributions, such as the Gaussian, and periodic signals such as the square or square tooth wave. 
			
			\begin{figure} [!ht]
				\centering
				\def\svgwidth{\columnwidth}
				\resizebox{16cm}{!}{\imginput{images/fourier-series-squarewave.pdf_tex}}
				\label{fig:fourier-series-squarewave}
				\caption{Fourier series of up to multiple elements n. We can see that as we increase the number elements which we include in our calculation, our fourier series becomes more square wave like but, the "gain" for including each successive element  decreases.}
			\end{figure}	
			
			. time frequency uncertainty
			results in
			expansion of time causes contraction in frequency
			This connection between time and frequency domain gives rise to an inherent duality. In quantum mechanics, this duality is also known as the energy time uncertainty principle. It is not obvious but, the energy is directly proportional to frequency as shown in equation $E=\hbar\omega=h\nu$. 
			
			This concept can also be applied to space and wavelength or kvector 
			ahh I'll come back to this after I ve taken some electrical engineering classes.
%-------------------------------------------------------------------------------%
%-------------------------------------------------------------------------------%	
	\section{Group velocity, Beat frequency and other related phase stuff}
%-------------------------------------------------------------------------------%
%-------------------------------------------------------------------------------%
	\section{Permutation}
		Permutations rearranges the order of an object. 
		ABC, ACB, BAC, BCA, CAB, CBA
		
		$\hat{P}_{ij} $ switches objects i and j and are called transpositions
		
		$\hat{P}_{AB} ABC = BAC$ 
		
		$\hat{P}_{ijk} ijk = jki$ known as cycles
		
		some permutations are equivalent to others, generally speaking $\hat{P}_{ik} = \hat{P}_{ki}$
		
		\begin{equation}
			\begin{pmatrix}
			1 & 2 & 3 & ... & n \\
			\alpha_1 & \alpha_2 & \alpha_3 & ... & \alpha_n &
			\end{pmatrix}
		\end{equation}
		
		the nucleis in an atom are are represented by
		
		\begin{equation}
			\begin{bmatrix}
			x_1, y_1, z_1, & x_2,  y_2,  z_2,& ... & x_n, y_n, z_n
			\end{bmatrix}
		\end{equation}
		
		The spaces between the sets of x,y, z represents the molecule 
		
		$x_i, y_i, z_i$ represents the coordinates of the ith molecules
		
		$P_{1n} 			
		\begin{bmatrix}
		x_1, y_1, z_1, & x_2,  y_2,  z_2,& ... & x_n, y_n, z_n
		\end{bmatrix} 
		=
		\begin{bmatrix}
		x_n, y_n, z_n & x_2,  y_2,  z_2,& ... & x_1, y_1, z_1
		\end{bmatrix} 
		$
		
		\begin{equation}
			\begin{bmatrix}
				r_1, & r_2, & ... & r_n
			\end{bmatrix}
		\end{equation}
		
		\begin{equation}
			\hat{P}_{123}
			\begin{bmatrix}
				r_1, & r_2,  & r_3
			\end{bmatrix}
			=
			\begin{bmatrix}
				r_3, & r_1,  & r_2
			\end{bmatrix}			
		\end{equation}
		
		\begin{equation}
		f(r_1,r_2,r_3)=r_1 + 2r_2 + 3r_3
		\end{equation}
		
		\begin{equation}
			\begin{split}
				\hat{P}_{123}f(r_1, r_2, r_3) = f^{(123)}(r_1, r_2, r_3) =r_3 + 2 r_1  + 3 r_2
			\end{split}
		\end{equation}
		
		\begin{equation}
			\begin{split}
				(123)(123)&=
				\begin{pmatrix}
					1 & 2 & 3 \\
					2 & 3 & 1
				\end{pmatrix} 
				\begin{pmatrix}
					1 & 2 & 3 \\
					2 & 3 & 1
				\end{pmatrix} 
				=
				\begin{pmatrix}
					2 & 3 & 1 \\
					3 & 1 & 2
				\end{pmatrix} 
				\begin{pmatrix}
					1 & 2 & 3 \\
					2 & 3 & 1
				\end{pmatrix} \\
				&=
				\begin{pmatrix}
					1 & 2 & 3 \\
					2 & 3 & 1 \\
					3 & 1 & 2 
				\end{pmatrix} 	
				=	
				\begin{pmatrix}
					1 & 2 & 3 \\
					3 & 1 & 2 
				\end{pmatrix}
				=
				(132)
			\end{split}
		\end{equation}
		
		(12345)=(12)(23)(34)(45) is even (even number of transpositions)
		
		(15236)(48)=(15)(52)(23)(36)(48) is odd (odd number of transpositions)
		
		E is called the identity operator (thank budha its not I, I already used 'I' for inertia tensor)
		
		$(123)(123)^-1 = E$
		
		The inverse of an cycle is done by representing it as its matrix form and switching the top and bottom rows
		
		\begin{equation}
		(abcd...xyz)^{-1}
		=
		\begin{pmatrix}
		abcd...xyz\\
		bcde...yza
		\end{pmatrix}^{-1}
		=
		\begin{pmatrix}
		bcde...yza\\
		abcd...xyz
		\end{pmatrix}
		=
		\begin{pmatrix}
		azy...edcb\\
		zyx...dcba
		\end{pmatrix}
		=(azyx...edcb)		
		\end{equation}
		
		\begin{enumerate}
			\item \(y_i=1\)
			\item \(y_i=2\)
		\end{enumerate}
		http://9gag.com/gag/ajqKpNw

\chapter{Experimental Techniques}
	\section{Modulation and Heterodyne Principle}
		\label{sec:Modulation and Heterodyne Principle}
		Modulation and heterodyning are techniques typically used in series to modify or transform a signal of some sort into another form that is more readily detectable and useful. For example, in the PDH locking technique, the carrier beam is modulated to generate sides bands which create a signal in the back reflection. The detector signal from the back reflection is then heterodyne with a local oscillator to produce the PDH error signal for locking.

		\subsection{Modulation}
			\label{subsec:Modulation}
			This section is to provide basic knowledge of modulation of a carrier waves and the heterodyne principle to aid in understanding of the complicated techniques used in this experiment, particular (so far) for Pound-Drever- Hall locking technique. Modulation and demodulation is a technique that is widely used in telecommunications, such as in cellphones and radio transmissions, to efficiently extract small of information stored in the modulation. Experimentally, this modulation data is stored in the spectral transition data for frequency modulation spectroscopy and NICE-OHM spectroscopy, the error signal of Pound-Drever-Hall locking technique and the more basic lock in amplification.
			
			The essential idea is to encode information in electric field of electromagnetic radiation either in the amplitude, frequency, or phase. This can be done by varying the power of the laser or by the use of nonlinear crystals. Here a simple plane wave traveling in some arbitrary direction. The magnitude of the modulation is define as the modulation depth/index which is denoted by $\beta$.
			
			\begin{equation}
				\tilde{E}(t)=\left(A_o\right)e^{i(\left[\omega_c\right] t + \left(\phi)\right)}
			\end{equation}
			
			Amplitude modulation involves varying the amplitude of wave with time such as with a sine function, $A(t)=\beta \sin{\Omega_m t}$ which will result in
			
			\begin{equation}
				\tilde{E}_{Amplitude}(t)=A_o\left(1+\beta \sin{\Omega_m t}\right) e^{i(\left[\omega_c\right] t + \left(\phi)\right)}
			\end{equation}
			
			Time varying the frequency/wavelength with a sine function will result frequency modulation
			
			\begin{equation}
				\tilde{E}_{frequency}(t)=\left(A\right) e^{i(\left[\omega_c (1+\beta \sin{\Omega_m t}\right)] t + \left(\phi)\right)}
			\end{equation}
			
			And finally phase modulation, the modulation technique of most interest in this thesis
			
			\begin{equation}
				\tilde{E}_{phase}(t)=\left(A\right)e^{i(\left[\omega_c \right] t + \left(\beta \sin{\Omega_m t})\right)}
			\end{equation}	
				
			The given equations are the time domain signals but the action of the modulation can be better described in their frequency domain. With the case for phase modulation, as long as the modulation index,$\beta$, is small, first order sidebands with $\omega \pm \Omega $ will be generated. Higher order sidebands containing interger multiples of $\pm \Omega$ are also present but are ignored as they contain higher orders of the modulation index which is assumed to be very small. More rigorous proofs can be seen in FM spectroscopy paper \cite{FMspec} and PDH locking technique paper \cite{PDH Intro}. In short, there are only 3 important frequencies after phase modulation to most techniques using phase modulation: the carrier frequency $\omega_c$ and two side bands $\omega_c$ 
			
			\begin{equation}
				\tilde{E}_{phase}(t)\approx E_o [e^{i\omega_c t}   +   \dfrac{\beta}{2} e^{i(\omega_c +\Omega_m)t}  -  \dfrac{\beta}{2} e^{i(\omega_c -\Omega_m)t}]
			\end{equation}
			
			Include picture with carrier frequency and side bands with frequency noted here
			This transformation from the time domain phase modulated to side can be described via a taylor expansion about the carrier frequency with modulation depth, $\beta$, assumed to very small. Alternatively, a more complicated calculation with a fourier transform can be done. I have yet to try the fourier transform.

		\subsection{Demodulation by Heterodyne Principle}
			\label{subsec:Demodulation by Heterodyne Principle}
			The Heterodyne principle is action of multiplying or mixing two sinusoidal waveforms which can be also be expressed as a sum of two sinusoidal waveforms whose frequencies are given by the sum and difference of the two mixed waveforms.\cite{MITModulation}. For example, if a photo-current from a photo-detector contains some sinusoidal components with a defined modulation frequnecy, then it can be extracted via the use of the Heterodyne principle. This is known as demoulation where we extract the information stored from the modulation. With some current signal containing two sinusoidal components of the same frequency, $\Omega_m$.
						
			\begin{equation}
				I_a(t)=I_1 (t) +I_2 (t) I_{noise}(t)= A_1 \cos{\Omega_m t} + A_2 \sin{\Omega_m t} +\sum_{\omega_i}^{} a_i \cos{(\omega_i t + \phi_i)}
			\end{equation}
			
			The \text{$I_1(t)=A_1 \cos{\Omega t}$} or \text{$I_2(t)=A_2 \sin{\Omega t}$} term can be extracted by demodulation via the use of the Heterodyne principal by mixing the photo-current signal with a sinusoidal current from a local oscillator of frequency $\Omega_{lo}$ with some phase $\phi_{lo}$. The noise terms will result in 0 signal after heterodyning so they will be ignored here on out.
		
			\begin{equation}
				\begin{split}
				I_b(t) &=I_a(t) \times B\cos(\Omega_{lo}t +\phi_{lo}) =(I_1 (t) +I_2 (t))*B\cos(\Omega_{lo}t +\phi_{lo}) \\
				& = A_1 B \cos{(\Omega_m t)}\cos(\Omega_{lo}t +\phi_{lo}) + A_2B \sin{(\Omega_m t)}\cos(\Omega_{lo}t +\phi_{lo}) \\
				& = \dfrac{A_1 B}{2} \left[ \cos{\left(\left(\Omega_m+\Omega_{lo}\right)t+\phi_{lo}\right)}  + \cos{\left( \left( \Omega_m - \Omega_{lo} \right)t +\phi_{lo}        \right)}                                    \right]\\
				& + \dfrac{A_2 B}{2} \left[ \sin{\left(\left(\Omega_m+\Omega_{lo}\right)t+\phi_{lo}\right)}  + \sin{\left( \left( \Omega_m - \Omega_{lo} \right)t -\phi_{lo}        \right)}                                    \right]\\
				\end{split}
			\end{equation}			 
		
			The goal now after mixing with the local oscillator is to rid of the plus combination of the two waveforms. With current in this form, this can be done with a low pass filter which will filter out high frequency components, ie the plus combination of the two sinusoidal waveforms.
		
			\begin{equation}
				I_c(t) = \dfrac{A_1 B}{2} \left[\cos{\left(\left(\Omega_m-\Omega_{lo} \right)t +\phi_{lo}\right)}\right]
				+\dfrac{A_2 B}{2}\left[\sin{\left(\left(\Omega_m-\Omega_{lo}\right)t-\phi_{lo}\right)}\right]
			\end{equation}					
			
			If we tune the local oscillator frequency so that frequency is equal to the modulation frequency and set the phase to be 0, ie $\Omega_{lo}=\Omega_m$ and $\phi_{lo}=0$, then we have obtained a dc component representation of the cosine term. This dc component can also be amplified for easier detection.
		
			\begin{equation}
				I_d(t) = \dfrac{A_1 B}{2}
			\end{equation}	
		
			If say, we instead set the phase of the local oscillator to be 90 degrees out of phase, ie $\cos(\Omega t + 90^o)=\sin(\Omega t)$, then we can extract the sine component of $I_a(t)$.
			
			In summary, by mixing a signal containing various waveforms of various frequency with a reference waveform of variable frequency and phase, the individual waveforms components can be experimentally extracted. To chemist who actually like with quantum mechanics, this is analagous to operating a general wave function with a state projection operator to obtain the component weight of the state.

	\section{Feed Back Control Theory}
		\label{sec:Feed Back Control Theory}
		asadfasdf talk a bit about feedback loops, include the book used for feedback control theory

	\section{Pound Drever Hall Locking Technique}
		\label{sec:Pound Drever Hall Locking Technique}
		
		\begin{figure} [!ht]
			\centering
			\def\svgwidth{\columnwidth}
			\resizebox{160mm}{!}{\imginput{images/PDH-setup.pdf_tex}}
			\label{fig:PDHSetup}
			\caption{a) is a cartoon showing a ray of light bouncing back and forth in a Fabry-Perot cavity while interacting with a sample. b) Illustrates a realistic interaction a resonating beam with a sample }
		\end{figure}
		
		The locking of a cavity refers to locking one of the cavity modes to the frequency of the laser beam so that optical power build may resonate and build up in optical power.
		The locking technique utilized is Pound Drever Hall (PDH) technique with high frequency modulation. This method of locking the lasing system was chosen due to its high sensitivity about the cavity resonance, fast response to frequency fluctuation and its ability to distinguish which side of the cavity resonance the laser frequency is relative to resonance \cite{PDH Intro}. The cost of utilizing this technique is the accompanied difficulty and complexity of this technique but a tight lock is required to minimize locking noise due to frequency modulation to amplitude modulation accompanied with the use of a cavity. In this technique, the incoming laser frequency beam is propagated through a crystal which modulates its phase, generating side bands with first order frequency differing by $\pm \Omega$ from the carrier wave.
		
		\begin{equation}
			\begin{split}
			E_{inc} & = E_o e^{i(\omega t + \beta\sin{\Omega t})}\\
			& \approx E_o \Big[J_o(\beta) e^{i\omega t} 
			+J_1(\beta)[e^{i(\omega +\Omega)t} +e^{i(\omega + \Omega)t}]\Big]
			\end{split}
		\end{equation}
		
		The approximation of the modulated wave is done by expansion of the Bessel functions with higher order terms ignored, or alternatively by the fourier transform of this time domain signal like in FTIR. This approximation is good when the modulation intensity is small, ie $\beta < 1$. What we are really interested in though is the intensity of the back reflection beam, not the cavity output beam, from the cavity since that is where the error signal is. At close to resonance and high frequency modulation, large $\Omega$, the carrier beam is assumed to be completely non reflecting, simplifying our power equation, $P_{\text{ref}}\propto |E_{ref}|^2$.
		
		\begin{equation}
			\begin{split}
			P_{ref} & = P_s\Big[|F(\omega +\Omega)|^2+|F(\omega -\Omega)|^2\Big]\\
			& + 2\sqrt{P_c P_s} \space \text{Im}  \big[ F(\omega)F^*(\omega - \Omega)-F^*(\omega) F(\omega - \Omega) \big]\sin (\Omega t)+(2\Omega terms)\\
			\end{split}
		\end{equation}
		
		\begin{equation}
			F(\omega)=\frac{E_\text{reflected}(\omega)}{E_\text{incoming}(\omega)}
		\end{equation}
		
		The dc and the 2$\Omega$ analog component signals are filter out. The leftover term is amplified and mixed with a local sinusoidal oscillator, with frequency $\Omega' = \Omega$, that can be varied with phase just like in lock in amplification \autoref{sec:Lock in Amplification}. The two signals are then mixed and after filtering of low frequency and dc component the final resulting error signal is
		
		\begin{equation}
			\text{Error Signal}= A_o \sqrt{P_c P_s} \text{Im}\big[F(\omega)F^*(\omega - \Omega)-F^*(\omega) F(\omega - \Omega) \big]\sin((\Omega + \Omega')t)
		\end{equation}
%
%		\begin{figure}[!ht]
%			\centering
%			\includegraphics[scale=0.55]{images/PDH_High_mod}
%			\caption{ The Pound-Drever Hall error signal of Low and High Frequency modulation respectively, Normalized Intensity vs $\omega /\Delta v_{\text{fsr}}$ \cite{PDH Intro} } 
%			\label{HFM PDH}
%		\end{figure}
		
		At a given free spectral  range, the intensity of the signal is 0 and quickly increases or decreases depending on which side the laser frequency drifts. 
		High frequency modulation is preferred over low frequency modulation since the slope is steeper with respect to resonance, allowing for a stronger and faster feed back signal to cancel out fluctuations in laser frequency drift. The signal generated to correct for such drifts was done by a PI controller.
		
		A more complete derivation and explanation can be found at \cite{PDH Intro}.
		
		\subsection{PI controller}
			\label{subsec:PI controller}
			A Proportion Integral controller is used to maintain a stable condition or state by use of an error signal. In the case of PDH locking, this controller locks the error signal to a given offset that corresponds to resonance. If the cavity is too far from resonance, the cavity becomes unlocked and the output voltage must be offset to obtain cavity resonance.
			PID controllers have a computer which will use an error signal to create an output signal to maintain resonance. The error signal will vary with time $e(t)$ due to noise.
			
			\begin{equation}
				u_{input}(e)=K_{Proportion}e(t)+K_{Integral} \int{e(t)dt} +K_{derivative} \frac{d}{dt}e(t)
			\end{equation}
			
			The controller that we use only calculates the the proportion and integral calculation hence PI controller and is missing the derivative component which is not nearly as important.
			This output signal intensity and polarity are then modified and sent to the cavity piezo to compensate for fast and slow fluctuations of the cavity length. In short, this output signal will then be used to maintain cavity resonance with the carrier wave from the OPO laser. 
			
			Some of the important variable functions of the controller PI corner knob, intensity of the proportion and integral signal, the servo mode and the 9db corner. 
			
			The PI corner knob sets the rate at which the proportion and integration calculations are done by the computer. Higher settings result in shorter delays in between output signals and more output signals in a given period. 
			
			There are amplifier knobs for the proportion and integral adjust the intensity of these component signals. Someone explain to me what this knob does please, I am desperate to know.
			Finally the Servo Mode mode affects the calculation the computers does for an input error signal for example, acquire is when the computer does no calculation when it receives the error signal. Pro stands for when the proportion calculation is sent and 6db, 6db+ are when the integration term begins I believe.
		
	\section{Lock in Amplification}
		\label{sec:Lock in Amplification}
		Lock-in amplification is a locking technique used to detect tiny alternating current signals. This technique allows for detection of signals with signal intensity in the range of nanovolts accompanied with the presence of large amounts of noise. A typical alternating current signal can be a sine wave, square wave or some other form of periodic signal.
		
		\begin{equation} \label{eq:signal}
			A_{\text{signal}} = A_{\text{sig}} \sin({\omega_{\text{sig}} t + \phi_{\text{sig}}}) 
		\end{equation}
		
		By multiplying the alternating current signal of interest with a reference signal of the same waveform, equal frequency, and phase, the signal of interest can be selected out from a large amount of background noise, as long as the noise level at this frequency is low. High noise at reference frequency results in a poor signal to noise ratio since we would end up picking up this noise as well. The usual solution is to bring the reference s
		
		\begin{equation}
			\label{eq:PSDsignal}
			\begin{split}
				A_{\text{PSD}} 
				= \frac{1}{2} A_{\text{sig}} A_{\text{ref}}\big[ \cos{[( \omega_\text{sig}-\omega_{\text{ref}}) + (\phi_\text{sig} - \phi_{\text{ref}})]} 
				+ \cos{[(\omega_\text{sig} + \omega_{\text{ref}}) + (\phi_\text{sig} + \phi_{\text{ref}})]}\big] 
			\end{split}
		\end{equation}
		
		With $\omega_{sig}=\omega_{ref}$ and by filtering of the high frequency term, $\omega_{sig}+\omega_{ref}$, we are left with new direct current signal. 
		
		\begin{equation}
			\label{eq:PSDDC}
			A_{\text{PSD}} = \dfrac{1}{2} A_{\text{sig}} A_{\text{ref}} \cos(\phi_{\text{sig}} - \phi_{\text{ref}})
		\end{equation}		
		
		\noindent
		From this direct current signal, \autoref{eq:PSDDC}, we know that it is proptional to our alternating current signal, \autoref{eq:signal}. If the signal is in the nanovolts region some more amplification of the raw locked in  data may be necessary before we can obtain our desired result. \cite{LIA}
		
		In laser spectroscopy, a physical chopper is used to chop the radiation source. This transforms the analog signal produced at the detector into a square wave. The frequency of the chopper is simultaneously sent into the Lock-in-Amplifier for use in the generation of the reference wave by the Lock-in-Amplification controller. The controller output the direct current signal, which is again, related to the signal detected at the detector by \autoref{eq:PSDDC}. 
		
		make LIA setup diagram
		
		Use of a chopper to physically transform the signal might seem highly counter intuitive at first since we are losing our signal but, it is actually highly sensitive. This high sensitivity is from the chopper and its controller \textbf{simultaneously} doing two things. The chopper physically turns the laser beam signal into a square wave while \textbf{simultaneously} communicating the chopping frequency to the Lock-in-Amplification controller. This means that there is essentially no difference between the reference frequency and chopping frequency.
		
\chapter{Experimental Setup}
	THIS IS GOING TO GET COMPLETELY REVISED TO REMOVE ALL THE TECHNICAL HARDWARE
	\section{Optical Setup}
		A diagram of the current setup it shown in figure \autoref{fig:ircease-setup}.
		The laser in this system is an tunable OPO laser with a fiber pump laser followed by a fiber amplification. The tuning of the laser is done by a crystal which splits an incoming beam into two beams of lower frequency. Both the amplification and splitting  of the beam are 2nd order linear processes. 
		The resulting output beams are 1\text{$\mu$}m from the pump laser, 3\text{$\mu$}m from the OPO process, and 0.7\text{$\mu$}m from the amplifier. There is one more frequency which corresponds to about 1.5 $\mu$m but it is not exiting the cavity. Only the 3\text{$\mu$}m  is used for spectroscopy. The other two frequencies are largely ignored once everything is aligned. 
		
		The lasing system and initial optics is represented by the box OPO Lasing System. Since it is quite complicated and dangerous. What is shown in figure \autoref{fig:ircease-setup} is a simple basic setup that satisfies the conditions for aligning a cavity, locking the cavity with Pound Drever Hall locking technique, and laser spectroscopy of a reference gas.
		
		A small portion of the beam is first reflected with the use of a wedge. This small portion is used for performing spectroscopy on the reference methane sample. The reflected beam passes through the methane KBr cell and is chopped. The modulated detection signal and reference signal of a few nanovolts is sent to a lock in amplifier which will output the IR laser absorption spectrum of methane.
		
		\begin{figure} [!ht]
			\centering
			\resizebox{160mm}{!}{\imginput{images/ircease-setup.pdf_tex}}
			\caption{}
			\label{fig:ircease-setup}
		\end{figure}
		
		The beam that transmits through the wedge is passed through a double lens system for modematching of the beam into the cavity. The beam is then passed through a polarizing beam attenuator which will split the beam into an S and P polarized beams. The beam that passes through is P polarized while S polarized beam exit through the escape port. The P polarized 3$\mu$m beam is then passed  through quarter wave plate turning the P polarized light into  right circularly polarized light (RCW) which is then coupled into the resonating cavity. Each time the beam hits a mirror, its circular polarization direction is switched. The resulting back reflections polarization is left circular polarized while the output of the cavity is right circularly polarized. The back reflection will exit through the input window and propagate through the wave plate causing it to now be S polarized. The back reflection now has an orthogonal polarization from the input beam Now the back reflection beam can be separated from input beam at the polarizing attenuator. The back reflection is primarily S polarized and is reflected off to the escape port which is sequentially measured.
		The tuning of our laser is controlled by Labview using a USB DAQ.
		
		Without the quarter wave plate and beam attenuator, the back reflection would follow the exact same path as the input beam making it impossible to measure without blocking the beam.
		
		The piezo stroke is controlled by a piezo controller followed by a function generator. The waveform applied to piezo is a saw teeth wave form with a sweep of 0-150V. This voltage sweep corresponds to a stroke distance of 1.5$\mu$. The output beam of the cavity is measured by another one of the liquid nitrogen dewar cool IR InSb detector. 
		
		Locking will be attempted once a sufficient signal to noise ratio is achieved. The target is 10:1 ratio but at our current progress, our ratio is at best 2:1. This is largely due to acoustic noise present in our setup due to the fan cooling the fibre amplifier system.

		\subsection{Laser Power}
			\label{subsection:Laser Power}
			The intensity spectrum of the laser has a periodic oscillation in intensity which may be due to the periodic polled structure within crystal in the OPO laser. There can also be large spikes in intensity followed by changes in wavenumber at the wavemeter due to the stability of differing modes resonating within the cavity varying during scanning. More time is spent looking for stable lasing conditions of the OPO laser before absorption spectrums are recorded and saved.
			
		\subsection{Cavity Setup}
			\label{subsection:Cavity Setup}
			The length of our cavity is 1.0m with two spherical symmetric mirrors with radius of curvature of 1.0m. At this cavity length, the free spectral range of our cavity is 0.10$cm^{-1}$. The cavity is symmetric confocal where $R_1=R_2=L_{cavity}$. The reflectivity coefficient at 3$\mu$m of the mirrors are 0.9998 to 0.9999 which correspond to a range of 8000 to 15000. The full width half max of our 3um beam is about 5 Angstroms which corresponds to 2500 finesse. Most of the broadening is most probably due to noise and difficulty in detection of such small stroke sizes and the lasing source being much more broadband then the cavity spectrum. The unexpected increased spread of the measured output relative from the theoretically calculated finesse is that the input beam is broadband while the \autoref{eq:Finess} from \autoref{sec:Optical Cavity and Resonance Properties} is based on the monochromatic wave.

	\section{Programs Used}
		\subsection{Labview}

		\begin{figure} [!ht]
			\centering
			\resizebox{160mm}{!}{\imginput{images/labview-program.pdf_tex}}
			\caption{The current state of the software}
			\label{fig:labview-program}
		\end{figure}

		The labview VI program uses event and state structures to synchronize the input and output voltage signals from a USB DAQ module. There are currently 4 input signals; 1 from the piezo controller, 1 from the power meter and 2 from the lock in amplifier. The Lock in amplifier has dual phase locking hence the two inputs. One of the input channel is used to control the peizo stroke of the pump laser to provide fine tuning of the pump laser for spectroscopy. The program will be upgraded to accept signals from at least 1 of the two liquid nitrogen dewar IR detector. 
		
		With event based programming, the program is compiled and the event structure will proceed to idle until it receives an event to notify itself to execute a specific state. There are 5 events, start scan, stop scan, save data, stop VI and the DO IT button all controlled by boolean data type. The start scan boolean will execute the do it button, then instrument initialization for data acquisition then finally followed saving of the array to a data file. The stop button will reverse the direction of the voltage ramp thus bring the wavelength back to the original value. The do it button executes a couple of calculations to estimate the time, max voltage etc of the scan to easily determine the length of a scan before deciding on the setting of a the scan. The stop VI terminates the event structure loop. Commands must be executed 1 at a time but they can be queued up. The program logs all the inputs and outputs into an tab delimited text file with headers name and numbering automatically generated based on the inputs at the control interface. The data file is then exported to IGOR for data processing with a script. 

		\subsection{Bristol Wavemeter}
			The wavemeter is a Bristol 621 IR wavemeter. It is used to obtain wavelength information. The provided stock Labview VI for controlling the wavemeter is inefficient and to ram intensive for the computer used. It has caused stability issues resulting in crashing of the computer. This is due to the wavemeter being designed to run on the language c. Instead the stock c language program is used to log the wavenumber of the beam. The program can be seen in figure \autoref{fig:labview-program}. Since the Labview and wavemeter program are not synchronized the wavemeter data is logged separately from the Labview data file. This synchronization problem is worked around with automatic data processing using the step function provided in the data file to fit the wavelength to the appropriate absorption signal. In the future, active X will be implemented to properly synchronized programs with Labview.
			
			\subsection{IGOR: Data Processing}
			
			Just some of raw and processed data outputted from the IGOR script.
			
			\begin{figure} [!ht]
				\centering
				\resizebox{160mm}{!}{\imginput{images/igor-process.pdf_tex}}
				\caption{}
				\label{fig:igor-process}
				
			\end{figure}
			
		\subsection{Python}
			A python script was created to simulate the cavity output signal as well as provide a means to quickly calculate theoretical free spectral range and broadness of of a cavity output spectrum. The input parameters are the cavity length, reflectivity of the wavelength and the wavelength of the beam. The script will be updated to have more features as the experiment is progressed and more advanced programming techniques are learned. This script is useful as it allows us to differentiate between 1$\mu m$ and 3$\mu m$ peaks.
			
%			\begin{figure} [H]
%				\centering
%				\resizebox{170mm}{!}{\input{images/pythonsimulation.pdf_tex}}
%				\caption{This is an image of two different compilations of the script. The reflectivity of the cavity mirror at 3$\mu m$ is about 0.9998-0.9999 as the manufactuors claim. The shape and distribution of the signal appear on a scope should be similar to the green line.}
%				\label{fig:pythonsimulation}
%			\end{figure}

%-------------------------------------------------------------------------------%
%-------------------------------------------------------------------------------%
%-------------------------------------------------------------------------------%
\chapter{Laser Alignment Tutorial}
	\section{Aligning the IR Cavity}
		\label{sec:Aligning the IR Cavity}
		The challenges of aligning the cavity and beam in this experiment is that the 3$\mu m$ beam is in the far IR and the length of the cavity is 1 meter. Far IR wavelengths are low in energy and are difficult to detect while larger cavity length have increased sensitivity to angle tilts. Since the project requires the use of both  FIR and long cavity length, extreme caution and patience is required in aligning IR and long cavity each on their own. The overall aligning process is about 4-5 hours to align the 3um with maximized resonance of the $TEM_{00}$ mode. Many of the procedures stated require many iterations to obtain strong resonance.
		
		The output laser beams consist of 0.700 \text{$\mu$m}, 1\text{$\mu$m}, and 3\text{$\mu$m}. The 0.7\text{$\mu$m} beam is ignored, the 1\text{$\mu$m} is used for initial rough alignment while the desired frequency of interest for rovibrational spectroscopy is 3\text{$\mu$m}. 
		Since the 3\text{$\mu$m} is visible only to specialized IR detectors, rough alignment must be done by use of another well colimated beam. Conveniently, the OPO laser by nature has another well collimated beam with a more easily detectable frequency, the 1\text{$\mu$m} pump beam. Though alignment 1\text{$\mu$m} is still difficult since it is invisible to the eye, it is high enough in energy to be viewed under specialized viewing scopes allowing alignment without the help of detectors. It is much more convient to align a beam that is at least visible with the use of a viewer.
		
		\begin{figure} [!ht]
			\centering
			\def\svgwidth{\columnwidth}
			\resizebox{130mm}{!}{\imginput{images/cav-align-improper.pdf_tex}}
			\caption{The red and blue beam are off center and coming at a tilted angle causing the back reflection beam to not accurately represent the mirror axis. Adjustments to the tilt of the mirrors are therefore unnecessary till the beam is sufficiently propagating through the center of both mirrors. 
				The process is to continually shift the beam pass the center then angle it towards the center at both input and output mirrors till the beam is propagating through the center of both mirrors.In both a) and b), the red beam must be shifted towards the blue then angle so that the beam hits the center of the window}
			\label{fig:cav-align-improper}
		\end{figure}
		
		The overall process is to align the cavity mirrors and 1\text{$\mu$m} beam to obtain resonance at atmopsheric pressure then proceed to vacuum the chamber followed by minor realignments with the 1$\mu m$ beam.  The process of alignment is explained in further down in the \autoref{subsec:Alignment of Vacuumed Chamber Cavity}.
		
		Once the $TEM_{00}$ is clearly resonanting, the 3\text{$\mu$m} beam is moved into the position of the 1\text{$\mu$m} beam. The 3\text{$\mu$m} should begin to resonate weakly in many modes as the beam is walked towards the cavity axis. This is done with a power meter designed for use with 3$\mu m$, printed circles, post it notes and tape. The power output of the beam is variable by controlling the amplificaion process, as such it can be increased high enough to leave burn marks on paper and post it notes. Large alignment is first done with the power meter and post it notes. The beam must pass through the quarter wave plate and beam attenuator/splitter. With the power meter infront of the wave plate, the beam 3$mu m$ is shifted towrads the center of the cavity and then angled in the oposite direction till the power meter reading is maxmized. The position of the beam is then roughly estimated with a post it note. This is repeated till the beam is close to the center then the printed circles are used to perform fine adjustments.  The trouble now is to determine which resonant peak is the $TEM_{00}$ mode. This was done visually by the use of a scope for the 1$\mu m$ since the $TEM_{00}$ is just a singular bright dot. The 3\text{$\mu$m} is not visible with a viewing scope so  the detector was instead moved vertically and horizontally with a micrometer stage. See \autoref{subsec:Mode Labelling}.
	
		\subsection{Alignment of 1m Vacuumed Chamber Cavity}
		\label{subsec:Alignment of Vacuumed Chamber Cavity}
		To align a cavity, the beam position must propagate roughly through the center of the mirrors and along the cavity axis. This is difficult since good cavity mirrors are expensive. The pair in our cavity together cost 2500 US dollars, thus interaction with the cavity mirrors must be minimized to avoid damaging them. Paper and tape on the other hand is cheap and plentiful. 
		
		For this experiment, the mirrors sit on a flange in a mirror holder with a circular opening that about the same size as the mirrors. Circles of the same size as the mirrors were printed and taped at the opening so that the centers of the circles matches the center of the mirrors. These printed circles are taped to both windows of the mirror holder to provide a means of scattering the beam at both ends. This allows for visual alignment since the location of the beam scattering from the paper is roughly where the beam is hitting relative the center of the mirror. 
		
		The beam is first aligned so that it hits the center of the output mirror (or out circle). Most likely the beam is most likely not going through the center of the input mirror, if it is rough alignment is done, its 99.99 \% of the time not if your cavity is 1.0m long. The input window is now placed back on the flange and the beam is shifted towards the center then angled so that it hits the center of the input window. Now the beam is slightly off center from the output mirror and the beam is again shifted towards the center then angled towards the center. Then the input mirror is removed and the beam is aligned again so that it hits the center of the input window. This process is repeated until the beam propagation axis is through the center of both mirrors. 
		
		Although the beam is going the center of the mirrors, it cannot resonate until it close to the cavity axis of the mirrors. Since the beam is roughly going through the center of the mirrors, the back reflections of both mirrors can be used to "move" the central axis of both axis. When both back reflection are aligned, the cavity axis is now well defined and is approximately defined by the back reflections. With the cavity now aligned with the input beam, and resonance should be detectable and close to the the $TEM_{00}$ mode. Now fine adjustments can be made to the mirrors and beam based on the mode that is strongly resonating within the cavity. Refer to \autoref{ssec:HowToAlignACavity} on how to tell higher modes can give information about types of misalignment's and modematching.
		
		In summary, the process is to use a beam that is well collimated and visible frequency either by the naked eye or viewing scope and align it so that goes through the center of both cavity mirrors. Once this is done, the back reflection of both mirrors now gives a rough estimate of where the cavity axis is located so the back reflections are aligned to input beam thus setting the cavity axis close to the propagation axis of the input beam. How this process is done will varies depending on the type of laser and constraints on the cavity mirrors.
		
		\begin{figure} [!ht]
			\centering
			\def\svgwidth{\columnwidth}
			\resizebox{130mm}{!}{\imginput{images/cav-align-proper.pdf_tex}}
			\caption{After sufficient iterations of shifting and changing tilt angles of the beam, the cavity axis, mirror axes, beam input and back reflection propagation axes should all be aligned allowing for resonance. }
			\label{fig:cav-align-proper}
		\end{figure}

		\subsection{Mode Labelling}
			\label{subsec:Mode Labelling}
			In this section, labeling of the $TEM_{00}$ mode will be discussed. For the 3$\mu$m beam, the peak corresponding to the $TEM_{00}$ mode was determined by checking the spatial signal intensity distribution. Visual means is not possible as 3$\mu$m is high only for detection with detectors. Each mode has its own transverse spatial distribution which are characterized by Gaussian distribution multiplied by polynomial factors. The result in every mode but the $TEM_{00}$ possessing nodes and can therefore be determined by spatially examining the intensity distribution of each peak and locating the resonance peak that only has 1 maximum. 
			
			This was done by sweeping the detector vertically and horizontally. From \autoref{fig:mode-label}, the peaks circled in black is the desired $TEM_{00}$ mode. This figure illustrates what was observed along a horizontal axis. The vertical was not shown as it detector is not on a vertical micrometer stage.
			
			\begin{figure} [!ht]
				\centering
				\def\svgwidth{\columnwidth}
				\resizebox{160mm}{!}{\imginput{images/mode-label.pdf_tex}}
				\caption{The target is to determine which node is the $TEM_{00}$, which is purely Gaussian. In this figure, only the horizontal distribution of the $TEM_{00}$ and maybe the $TEM_{10}$ is shown. If the adjacent mode is the $TEM_{10}$ mode, than there is mis alignment along the horizontal direction.}
				\label{fig:mode-label}
			\end{figure}
			
			Only the horizontal displacement of the detector is shown since the detect is not mounted on a vertical micrometer stage.						
%-------------------------------------------------------------------------------%
%-------------------------------------------------------------------------------%
%-------------------------------------------------------------------------------%
						
%%%%%BIBLIOGRAPHY%%%%%
\input{bibliography}		
\end{document}

